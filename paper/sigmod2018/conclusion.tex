\section{Conclusions and Future Work}\label{sec:conclusion}
We have shown how to improve transaction performance in an OCC system by integrating batching and reordering into the system architecture at storage and validator. Batching allows the reordering of requests and reduces the number of aborts. We have formulated validator reordering as the problem of finding the minimal feedback vertex set (FVS) in a directed graph, and we have proposed two greedy algorithms for finding a FVS that are flexible and that perform well in practice. We have carried out an extensive experimental study in a main memory transaction processing prototype, as well as implemented a client side solution for a commercial database. Our experiments show that there is a sweet spot for the batch size for the best balance between latency and the flexibility of reordering. We have also investigated the impact of storage and validator batching on system performance. While both storage and validator batching consistently improve the throughput, validator reordering significantly reduces the latency profiles. We further proposed a parallel validator design. Our experiments showed that parallel reordering has improved both throughput and latency.

In future work, we plan to explore more sophisticated batch creation techniques, as well as systematically investigating adaptive batching to intelligently enable batching and adjust batch sizes for best system performance. As recent open source transaction processing kernels offer impressive
read / write throughput, we also want to explore the opportunity to incorporate our techniques into these kernels.
