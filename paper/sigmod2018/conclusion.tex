\section{Conclusions and Future Work}\label{sec:conclusion}
We have shown how to improve transaction performance in an OCC system by integrating batching and reordering into the system architecture at storage and validator. Batching allows the reordering of requests and reduces the number of aborts. We have formulated validator reordering as the problem of finding the minimal feedback vertex set (FVS) in a directed graph, and we have proposed two greedy algorithms for finding a FVS that are flexible and that perform well in practice. We showed that our algorithms can integrate different policies into the reordering to optimize for alternative objectives, such as low tail latency. Moreover, we proposed a parallel validator design to reduce the overhead of reordering. We have carried out an extensive experimental study in a main memory transaction processing prototype, as well as on top of a commercial database. Our experiments show that both storage and validator batching consistently improve throughput, and validator reordering significantly reduces latency profiles. We also demonstrated how we optimize for low tail latency with alternative policies. In addition, we showed that parallel reordering effectively improved throughput and reduced latency.

In future work, we plan to explore more sophisticated batch creation techniques. Since we observed a sweet spot for batch size in our experiment, we want to systematically investigate adaptive batching that intelligently enables batching and adjusts batch sizes to achieve the best performance. As recent open source OLTP kernels offer impressive
throughput, we also plan to explore the opportunity to incorporate our techniques into those kernels.
