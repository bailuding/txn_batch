\documentclass[sigconf]{acmart}

\usepackage{booktabs} % For formal tables


\usepackage{graphicx}
\usepackage{balance}  % for  \balance command ON LAST PAGE  (only there!)

\usepackage[linesnumbered]{algorithm2e}
\usepackage{caption}
\usepackage{subcaption}
\captionsetup[subfigure]{labelformat = parens, labelsep = space, font = small}
\usepackage{enumitem}

%\usepackage{algorithm}
%\usepackage{algpseudocode}

% To get colored comments
\usepackage{color}  
\newcommand{\johannes}[1]{{\color{blue} Johannes: [{#1}]}}
\newcommand{\lucja}[1]{{\color{red} Lucja: [{#1}]}}
\newcommand{\bailu}[1]{{\color{cyan} Bailu: [{#1}]}}
\newcommand{\magda}[1]{{\color{magenta} Magda: [{#1}]}}

\newcommand{\eat}[1]{} % comment out blocks of text

% Copyright
%\setcopyright{none}
%\setcopyright{acmcopyright}
%\setcopyright{acmlicensed}
\setcopyright{rightsretained}
%\setcopyright{usgov}
%\setcopyright{usgovmixed}
%\setcopyright{cagov}
%\setcopyright{cagovmixed}


% DOI
%\acmDOI{10.475/123_4}

% ISBN
%\acmISBN{123-4567-24-567/08/06}

%Conference
%\acmConference[WOODSTOCK'97]{ACM Woodstock conference}{July 1997}{El
%	Paso, Texas USA} 
%\acmYear{1997}
%\copyrightyear{2016}

%\acmPrice{15.00}

\begin{document}

% ****************** TITLE ****************************************

\title{Improving OCC Performance Through Transaction Batching and Operation Reordering}


%\numberofauthors{4}

%\author{
% You can go ahead and credit any number of authors here,
% e.g. one 'row of three' or two rows (consisting of one row of three
% and a second row of one, two or three).
%
% The command \alignauthor (no curly braces needed) should
% precede each author name, affiliation/snail-mail address and
% e-mail address. Additionally, tag each line of
% affiliation/address with \affaddr, and tag the
% e-mail address with \email.
%
% 1st. author
%\alignauthor Bailu Ding \\
%	\affaddr{Cornell University, Microsoft Research}\\
%    \email{badin@microsoft.com}
% 2nd. author
%\alignauthor Lucja Kot \\
%	\affaddr{Cornell University}\\
%    \email{lucja@cs.cornell.edu}
% 3rd. author
%\alignauthor Magdalena Balazinska\\
%	\affaddr{University of Washington}\\
%	\affaddr{Seattle, Washington, USA}\\
%    \email{magda@cs.washington.edu}
%\and
%\alignauthor Johannes Gehrke\\
%	\affaddr{Microsoft Corporation}\\
%    \email{johannes@microsoft.com}
%}

\maketitle

\begin{abstract}
OLTP systems can often improve throughput by \emph{batching} transactions and processing them as a group. Batching has been used for optimizations such as message packing and group commits; however, there is little research on the benefits of a holistic approach to batching across a transaction's entire life cycle. 

In this paper, we present an OLTP system based on OCC that incorporates batching at multiple stages of transaction execution. Storage batching enables reordering of transaction reads and writes at the storage layer, and validator batching enables reordering of transactions before validation. We formalize the problem of validator transaction reordering, and we propose several algorithms and policies to solve this problem. We explain how validation can be parallelized for better performance. We carry out an in-depth experimental evaluation of the impact of storage and validator batching, and we show that our techniques can significantly increase throughput and reduce latency. We show how different batching algorithms perform, how parallelism in validator reordering helps, how storage and validator batching interact with each other, and how they improve system-wide and individual transaction performance.
\end{abstract}

\section{Introduction}\label{sec:intro}

% batching is important
Transaction processing is a fundamental aspect of database functionality, and improving OLTP system performance has long been a key research goal in our community. It is well-known that the throughput of OLTP systems can be increased through \emph{batching}-based optimizations, whereby some component buffers a number of transactions or requests as they arrive and processes them as a group.

% batching for obvious reason: low level
Batching can improve system performance for several reasons. First, it increases the efficiency of communication by packing messages~\cite{ding2015centiman,friedman1997packing}. Second, it amortizes the cost of system calls by condensing multiple requests into a single one, as in group commit~\cite{debrabant2013anti,hagmann1987reimplementing}. Third, it reduces the number of requests by discarding duplicate or stale requests, such as writes to the same record~\cite{faleiro2014lazy}. However, all of those are local optimizations based on low-level techniques.

% our proposal at a high level: batching at higher level for OCC 
We propose an OLTP system design that embraces batching as a core design principle at multiple stages of a transaction's execution. In particular, we explore the benefits of batching in optimistic concurrency control (OCC) to reduce conflicts~\cite{kung81tods}, and thus improve both system-wide and individual transaction performance. OCC is a popular concurrency control protocol due to its low overhead in low-contention settings~\cite{adya97podc, baker11cidr, bernstein2015optimizing,bernstein11cidr, bernstein11vldb, corbett12osdi,warp, patterson12vldb,peng10osdi}. However, it wastes resources when conflicts are frequent~\cite{agrawal1987concurrency}. We show how batching reduces the number of conflicts and how we can thus use OCC even with higher-contention workloads.


Figure~\ref{fig:occ_arch} shows a distributed OCC-based transaction processing system with centralized validation. The system consists of three components: processors, storage nodes, and a single validator. External clients issue transactions to the system. On arrival into the system, each transaction is assigned to a processor and enters its \emph{read} phase. The processor sends read requests to the storage, executes the transaction, and performs writes to a local workspace. After it has processed the transaction, it sends information about the transaction's reads and writes to the validator. 

The transaction now enters the \emph{validation} phase. In OCC with \emph{backward validation}, the validator checks whether the transaction conflicts with any previously committed transactions, and makes a ``success'' or ``failure'' validation decision. 
%jg01: We need a citation after ``decision.''
One example of a conflict that would cause validation to fail is a \emph{stale read}. Suppose a transaction $t$ reads an object, and a second transaction $t'$ writes to the same object after $t$'s read. If $t'$ commits before $t$, $t$ has a conflict, since it should have read the update $t'$ made to the object but it did not. Thus $t$ must fail validation. 


If a transaction passes validation, the processor sends its writes to the storage; this is the \emph{write} phase. Otherwise, the processor aborts and restarts the transaction.

% why OCC for batching
The architecture of OCC with backward validation presents unique opportunities for batching because transactions are only serialized prior to commit. 
% batching at processor
There are three obvious times and locations to apply batching. The first is the processor in the transactions' read phase, where transaction requests can be batched before execution. There is recent work in the context of locking-based protocols that  batches transactions and serializes them before execution to reduce overhead~\cite{faleiro2014rethinking,mu2014extracting,thomson2012calvin}; these techniques could be adapted and applied in OCC as well.

% batching at validator
The second place we can batch is the validator, for transactions in the validation phase. If the validator buffers the validation requests as they arrive, it has the flexibility to choose a validation order. This allows the validator to reduce the number of conflicts and aborts. Recall our previous example of a validation failure. A transaction $t$ reads an object, and a second transaction $t'$ writes to the same object after $t$'s read. If $t'$ arrives at the validator before $t$ and commits, $t$ must fail since it should have read the value written by $t'$ but it didn't. Instead, with batching, if transactions $t$ and $t'$ are both in the same batch waiting for validation, we can choose to serialize $t$ before $t'$. Thus, we avoid conflicts and both transactions can commit.

% batching at storage
Third, the system can perform batching at the storage level. This affects already-validated transactions in their write phase as well as transactions still in their read phase. The storage can buffer read and write requests into batches as they arrive. If a batch contains multiple read and write requests for the same object, the system can apply all the writes first, in the serialization order. Next, it can process the reads. Prioritizing writes over reads is always optimal in the sense that it reduces the number of aborts as much as possible. This is because OCC reads come from uncommitted transactions, while writes come from validated transactions that will commit soon. Thus if the storage has both a pending read and a pending write on the same object, but schedules the read before the write, the reading transaction will see a stale value and is guaranteed to fail validation. 

% drawback of batching
%Batching at these three levels in the system reduces aborts due to conflicts, and thus can increase throughput.\eat{However, it may also increase latency.}

\begin{figure}[t]
 \centering
 \includegraphics[width=0.9\columnwidth]{figures/OCCArchitecture.pdf}
 \vspace{-.5em}
 \caption{OCC System Architecture}
% \vspace{-1.5em}
 \label{fig:occ_arch}
\end{figure}


% contribution
{\bf Contributions of this Paper.}
In this paper, we explore in-depth the benefits of transaction batching in OCC with backward validation. Since there is substantial existing work on processor batching, we focus on storage and validator batching, which have received little research focus so far.
% contribution 1: system design
Our first contribution is an enhanced OCC system architecture that creatively includes batching at the storage and the validator levels. We analyze the reasons for transaction conflicts and aborts at each stage of a transaction's life cycle, and develop techniques to reduce these aborts through batching and operation reordering.
% contribution 2: validator reordering algorithms
Our second contribution is an optimal algorithm for storage batching and approximate algorithms for validator batching. We show that optimal transaction reordering within a validator batch is NP-hard, and we describe two classes of greedy algorithms. These algorithms produce transaction orderings that are close to the optimal solution, and they are sufficiently fast for practical use. Since the overhead of validator reordering increases transaction latency, it may hurt the end-to-end throughput of the system even it reduces the number of aborts within a batch. Our algorithms explore this trade-off and show how we can achieve low abort rates with little overhead. We also extend our algorithms to weighted versions that can incorporate features such as transaction priorities.\eat{ Thus our second contribution actually enables storage and validator batching in a real system.}


\eat{
As mentioned before, the overhead of validator reordering increases transaction latency. Thus, even as validator reordering reduces the number of conflicts, it may hurt the end-to-end throughput of the system. Our two classes of greedy algorithms explore this trade-off. Additionally, we extend our algorithms to weighted versions that can incorporate features such as transaction priorities.
}

% contribution 3: experiment
Our third contribution is a detailed experimental study of the impact of storage and validator batching in a prototype system. Our results show that batching always increases transaction throughput and, surprisingly, also reduces transaction latency. For workloads with high data contention, batching and reordering can improve the throughput by up to a factor of 2.4 and can reduce the average transaction latency down to 21\% for both micro-benchmark and the Small Bank Benchmark~\cite{alomari2008icde}.

%\eat{
% paper organization
The remainder of the paper is organized as follows. In Section~\ref{sec:background}  we review OCC with backward validation. In Section~\ref{sec:overview} we discuss challenges, opportunities and techniques for storage and validator batching. In Section~\ref{sec:validator_reordering}, we introduce our algorithms for intra-batch transaction reordering at the validator. In Section~\ref{sec:experiments}, we present an extensive experimental study of batching in our system. We discuss related work in Section~\ref{sec:relwork} and conclude in Section~\ref{sec:conclusion}.
%}


\section{OCC Background}\label{sec:background}

% backward occ
In this paper, we use the term ``optimistic concurrency control'' to refer to the classical backward validation based protocol introduced in~\cite{kung81tods}. As explained in the introduction, this is a protocol where every transaction goes through three execution phases. First comes a \emph{read} phase, where the transaction reads data from the storage and executes while making writes to a private ``scratch workspace''. Then the transaction enters a \emph{validation} phase. If validation is successful, the transaction enters the \emph{write} phase, when its writes are installed in the storage. 

% details for validation
OCC validation enforces serializability; we only allow a transaction to pass validation if it can be serialized after all transactions that have already committed. The validator assigns each transaction a number (timestamp) and ensures that transactions are serialized in timestamp order.

To validate a transaction, we need to examine its reads and writes, and compare them to the writes of previously committed transactions. For a given transaction, we call the set of objects it has read its \emph{read set} and the set of objects it has written its \emph{write set}. When a transaction enters validation, its read set and write set are known, as are write sets of all previously committed transactions. For a transaction with timestamp $i$, we denote its read set by $RS(i)$ and its write set by $WS(i)$.

% the scope of validation
When validating a transaction with timestamp $j$ (transaction $T(j)$), the validator needs to check for conflicts with any transaction with timestamp $i$ (transaction $T(i)$) that have already committed (so $i < j$) and that overlapped temporally with $T(j)$, i.e. $T(i)$ hadn't committed when $T(j)$ started. 
% criterion for conflicts
$T(j)$ can be serialized after such a transaction $T(i)$ if one of the following conditions holds:
\begin{itemize}
\item $T(i)$ completed its write phase before $T(j)$ started its write phase, and $WS(i) \cap RS(j) = \emptyset$, or
\item $T(i)$ completed its read phase before $T(j)$ started its write phase, and $WS(i) \cap RS(j) = \emptyset$ as well as $WS(i) \cap WS(j) = \emptyset$
\end{itemize}

If $T(i)$ and $T(j)$ overlap temporally, and $WS(i) \cap RS(j) \neq \emptyset$, we say there is a \emph{read-write dependency} from $T(j)$ to $T(i)$. Intuitively, if there is such a dependency, $T(j)$ cannot be serialized after $T(i)$. In addition, we must ensure the writes of the two transactions are installed in the correct order to maintain consistency in the storage. If they write to the same object, the updates must be applied in their serialization order. In the original OCC~\cite{kung81tods}, this is ensured by putting the validation phase and the write phase into a critical section.

The batching techniques we study in this paper are applicable to most OCC systems; however, we make certain assumptions about the implementation of OCC validation, for simplicity and concreteness of our experimental setup. We clarify those assumptions now.

Given that our architecture (Figure~\ref{fig:occ_arch}) has a single validator, we assume the validation phase is sequential. We do not require the write phase to be in a critical section; this means that write requests may arrive at the storage out of order. We deal with this using a \emph{versioned} datastore; every item in the datastore is versioned and every write request is tagged with a version number equal to the writing transaction's timestamp. If the datastore receives a write request with version (timestamp) $i$ and a higher-numbered version $j > i$ already exists for the object, the write request is ignored. 

The above discussion implies that validation only needs to check, for every transaction $T(j)$ and all transactions $T(i)$ overlapping temporally with $T(j)$, where $i<j$, that $WS(i) \cap RS(j) = \emptyset$. Data versioning provides an easy way to determine whether $T(j)$ overlapped temporally with $T(i)$; every time a transaction performs a read, we tag the read with the version of the item that was read. If $T(j)$'s read set contains an item $X$ and the read saw version $k$, the validator only needs to check the write sets of all $T(i)$ with $k < i < j$ to see whether they contain $X$. This use of versioning in validation is identical to the one described in~\cite{ding2015centiman}.

\section{Batching}\label{sec:overview}

%In this section, we introduce our novel ideas of batching at the storage and the validator in an OCC system, and we formulate the resulting algorithmic problems. 

\emph{Batching} involves buffering a number of operations as they arrive at some component of the system and processing them as a group. 
Given a batch, we run a lightweight algorithm that analyzes the batch and then reorders the operations in the batch in order to reduce aborts. We will introduce two types of batching: Validator batching and storage batching.


\subsection{Validator Batching}\label{subsec:overview:validator}

In validator batching, we buffer and batch transaction validation requests at the validator as they arrive.
We can first make a conceptual distinction between two types of transaction aborts: intra-batch and inter-batch aborts. Assume $T(j)$ abort due to its conflict with $T(i)$. If $T(i)$ and $T(j)$ are in the same batch, we call the resulting abort of $T(j)$ an \emph{intra-batch abort}; otherwise, we call it an \emph{inter-batch abort}.
The optimal strategy for reducing inter-batch aborts is to ensure that transactions accessing the same objects are always batched together. That is, to cluster them based on access patterns, since  the validator cannot do anything about conflicting inter-batch transactions. 
There has been a lot of research on data clustering either online or offline~\cite{elmore2015squall, pavlo2012skew}, especially for fine-grained data partitioning. Any of these techniques can be used for validator batch creation.
%However, clustering transactions into batches based on access patterns may increase the number of intra-batch aborts.

Once a batch is collected, the validator can reduce intra-batch aborts by choosing a good validation (and thus resulting serialization) order.
\eat{The validator may not be able to eliminate all intra-batch aborts, since it cannot actually delay the execution of a transaction. Rather, it is presented with read and write sets of transactions that have already run. }
Such intra-batch transaction reordering can be done with several goals in mind. We can simply minimize intra-batch aborts, i.e. we maximize the number of transactions in each batch that commit. However, we may also want to prioritize certain transactions to have a greater chance of committing. For example, if we want to reduce the transactions' tail latency, we can increase a transaction's priority every time it has to abort and restart. Priorities could also be related to external factors, e.g., a transaction's monetary value or an external, application-defined transaction priority. 
%These choices suggest a range of possible policies; we explore them in the next section.

We now define the problem of {\em intra-batch validator reordering of transactions (IBVR)} more formally. A \emph{batch} $B$ is a set of transactions to be validated. We assume all transactions $t \in B$ are \emph{viable}, that is, no $t \in B$ conflicts with a committed transaction. If there are non-viable transactions in $B$, they can be removed in preprocessing, as they must always abort.
Given $B$, the goal of IBVR is to find a $B' \subseteq B$ of transactions that must abort due to intra-batch aborts. IBVR must also find a 
%strict (i.e. asymmetric) 
total order $\prec$ on $B \setminus B'$ that \emph{respects all read-write dependencies}; that is, for $t,t'\in B \setminus B'$, if $t \prec t'$, then there is no read-write dependency from $t'$ to $t$.
The validator processes each batch by running IBVR to identify $B'$ and $\prec$, aborting all the transactions in $B'$  and validating the transaction in $B \setminus B'$  in the order $\prec$. By the constraint we gave on $\prec$ above, and from the discussion in Section~\ref{sec:background}, $\prec$ is guaranteed to be a valid serialization order that allows all transactions in $B \setminus B'$  to commit.

%Note that there is always a trivial solution to any IBVR instance: we can choose $B'$ to be all the elements of $B$ except one, so that $B \setminus B'$ has cardinality one. Such solutions are not useful; therefore, every instance of IBVR is associated with an \emph{objective function} on $B'$, and the goal is to find a $B'$ that maximizes the objective function. An example objective function could be the size of $B'$ (smaller is better), or a more complex function that takes into account transaction weights (e.g. priorities).

%Given $B$, the goal of IBVR is to find a $B' \subseteq B$ of transactions that must abort due to intra-batch aborts, and a commit order $Q$ for the remaining transactions that respects all read-write dependencies. The validator processes each batch by running IBVR to identify $B'$ and $Q$, aborting all the transactions in $B'$ and committing remaining transactions in the order of $Q$.

There is always a trivial solution to any IBVR instance: aborting every transaction but one. Such solutions are not useful; therefore, every instance of IBVR is associated with an \emph{objective function} on $B'$, and the goal is to find a $B'$ that maximizes the objective function. An example objective function could be the size of $B'$ (smaller is better), or a more complex function that takes into account transaction priorities.






\subsection{Storage Batching}
At the storage layer, the optimal batching strategy is simple. We buffer a certain number of read and write requests into batches. The only way to reduce inter-batch aborts is to increase the batch size.\eat{, but we have more flexibility for intra-batch optimizations.} To reduce intra-batch aborts, we propose the following simple algorithm, which is easy to prove to be optimal given a batching:  When a batch is ready (either when there are enough requests or a timeout is reached), for each object we apply the highest-version write request on that object. It is safe to discard all other writes on that object as explained in Section~\ref{sec:background}. Next, we handle all the read requests for the same object. This strategy is optimal for reducing intra-batch aborts, as it ensures that all available writes by committed transactions are applied to all objects before we handle any read requests on these objects. 





\eat{

At the validator, we can change the assignment of timestamps and the resulting serialization order of transactions in order to avoid aborts. The overhead of the reordering impacts transaction latency, so the reordering algorithms must be fast.
For example, we can use operation batching at the storage layer to analyze the sequence of reads and writes and reorder them to avoid some transaction aborts. 


 The type of operations at each component in the system are different. 
 


The goal of batching is to reduce the number of conflicts, 
%by reordering the operations in each batch, 
thus decreasing abort rate, increasing throughput and lowering atency. As explained in Section~\ref{sec:background}, aborts happen due to read-write dependencies involving a committed transaction $T(i)$ that wrote to some object, and a later transaction $T(j)$ that read the same object but did not see $T(i)$'s update. 




In this paper, we propose two types of batching

The server components -- the storage and the validator -- aim to reduce both intra-batch and inter-batch aborts. The storage batches the read and write requests, and the validator batches the validation requests.

Given a batch, the number of intra-batch aborts is determined by the way we order and interleave the requests in processing. Thus the main optimization strategy is \emph{reordering}. 
Neither inter-batch nor intra-batch conflicts can be completely eliminated by reordering. A transaction that passes validation but gets ``stuck'' before writing to the storage will conflict with all subsequent transactions that read the same object, causing unavoidable inter-batch conflicts. Two transactions in the same batch which read and write the same object cannot both commit, thus causing an unavoidable intra-batch conflict.

 %Note that not all components in the system need to batch transactions in the same way.


\eat{
To reduce the number of inter-batch aborts, the system can divide the transactions into batches in a careful manner; it can also privilege writes over reads to allow reads to see fresher values. However, at all times in the system, there are multiple batches in-flight. 

No component can eliminate inter-batch aborts completely and unilaterally. The worst case is a slow transaction that writes an object and passes validation, but whose write gets ``stuck'' in the system and doesn't make it to the storage for a long time. This transaction will conflict with -- and cause inter-batch aborts of -- all subsequent transactions that read the same object, until the write is actually applied.

We now explain the specific batching strategies that are available at the storage and the validator.}
}%\eat
\section{Reordering}

At the validator, we reorder the transactions based on max-commit, priority, or fairness policy.

We say an ordering of the transactions is \textbf{\emph{sound}} if the transactions can commit in that order without violation of the isolation levels.

Assume we run the transactions at serializability \magda{I assume that we will discuss the different levels of isolation somewhere in this section.}, an ordering is sound if and only if there is no read-after-write dependencies from a transaction $T_{later}$ that is later in the sequence to a transaction $T_{early}$ that is earlier in the sequence, i.e. $read\_set(T_{later})\cap write\_set(T_{early})\not\emptyset$; otherwise, $T_{later}$ will read stale versions and will be conflicting with $T_{early}$. 
\magda{What about write-write conflicts?}

We first formulate the soundness of the problem as the feedback vertex
set problem \magda{cite a source to show that this is a known type of
  problem} in a directed graph:

\paragraph{Feedback Vertex Set} Given a directed graph, a feedback vertex set of a graph is a set of vertices whose deletion makes the graph acyclic.

\paragraph{Soundness} Given a batch of transactions and create a directed graph $G$ with transactions as nodes $N$ and read-after-write dependencies as edges, there exists a sound ordering corresponding to a feedback vertex set. 

\paragraph{Finding a Feedback Vertex Set is Equivalent to Soundness} 
We can construct a sound ordering with the feedback vertex set of the dependency graph. 
Since the graph is acyclic, we can iteratively choose a vertex that does have a forward dependency, i.e. no out edges.
On the other hand, if the ordering is sound, there is no dependency edge from a later transaction in the sequence. 
Thus, there is no cycle in the directed graph. 

\paragraph{Greedy Algorithm for Max-Commit Policy}
Finding an order with max-commit policy can be directly reduced to the problem of finding the minimal feedback vertex set. Unfortunately, this is listed in the first set of the NP-complete problems~\cite{karp1972reducibility}.

As mentioned in Section~\ref{design:subsec:batching}, it is important to come up with a good ordering rapidly so that the decreasing in in-batch aborts will not be offset by the increasing in pre-aborts.  We come up with a greedy algorithm to find a sound ordering, attempting to maximize the number of commits.
\magda{Maybe we should compare the performance of exhaustive search and greedy?}

Given a batch of transactions at hand, we first construct the dependency graph
as shown in Algorithm~\ref{reorder:alg:graph}. We construct the graph by first creating a hash table to cache all the updates in this batch. The running time of the construction is $O(|V|+|E|)$, where $|V|$ is the number of nodes and $|E|$ is the number of edges.

Then we remove all the transactions in the batch that conflict with committed transactions (Algorithm~\ref{reorder:alg:pre_abort}), i.e. pre-aborts. These transactions are bound to abort and we will not consider them in the reordering.

Algorithm~\ref{reorder:alg:greedy} shows the core of the greedy algorithm. 
The idea is to first construct a heap with all the remaining transactions that are not pre-aborted, and then to iteratively choose the transaction that will incur the least number of aborts for the unscheduled transactions until the heap becomes empty (line 6).
After choosing a transaction $t$ (line 7), we have to abort any unscheduled transaction $t'$ that has a read-after-write dependency on $t$ (line 8-15), i.e. $t'$ should have read the update from $t$. 
Then we clean up all the dependencies on $t$ (line 16), add $t$ to the ordered sequence (line 17), and get ready to choose the next transaction.
The running time of the core greedy algorithm is $O(|V|log(|V|))$.

Inside the heap, we order the transactions by the number of aborts that will be incurred if we schedule a transaction as the next transaction, as shown in Algorithm~\ref{reorder:alg:max_commit}. 

The total running time of this greedy strategy is $O(|E|+|V|+|V|log(|V|))$. 
which becomes $O(|V|log(|V|))$ if $|E|=O(C|V|)$ for some constant $C$.

\paragraph{Weighted Greedy Algorithm for Other Policies}
We can extend the greedy algorithm to other policies by adding some weight in the comparison function of the heap.

For priority policy, as shown in Algorithm~\ref{reorder:alg:priority}, we can first check whether the two transactions have the same priority (line 1-3). 
If not, we choose the one with higher priority; otherwise, we choose the one that will cause the least number of aborts (line 4-6).

Similarly, we implement the fairness policy as shown in Algorithm~\ref{reorder:alg:fairness}. 
If it is not fair to schedule a transaction $t$ before a transaction $t'$, we place $t'$ before $t$ (line 1-3); otherwise, we choose a transaction that will cause the least number of aborts (line 4-6).

\begin{algorithm}
\KwData{A batch of $T$ transactions with reads and writes}
\KwResult{A read-after-write depedency graph $G$}
\tcp{construct hash table of updates}
$hash \gets \emptyset$\;
\For{$t \in T$} {
	\For{$w \in t.write\_set$} {
		$ids = hash.get(w)$\;
		$hash.put(w, ids.add(t.id))$\;
	}
}
\tcp{construct dependency graph}
$G\gets \emptyset$\;
\For{$t \in T$} {
	\For {$r \in t.read\_set$} {
		$ids = hash.get(r)$\;
		\For {$id \in ids$} {
			$G.add\_edge(t.id, id)$\;
		}
	}
}
\Return{$G$}\;
\caption{Construct the read-after-write dependency graph}
\label{reorder:alg:graph}
\end{algorithm}

\begin{algorithm}
\KwData{A read-after-write dependency graph $G$ and a batch $T$ of transactions}
\KwResult{A set $S$ of transactions that do not conflict with committed transactions}
$S \gets \emptyset$\;
\For {$v \in G$} {
	$t\gets T.get(v)$\;
	\If {$validate(t) == COMMIT$} {
		$t.dec = TBD$\;
		$S.add(t)$\;
	}
	\Else {
		$t.dec = PRE\_ABORT$\;
	}
}
\Return{$S$}\;
\caption{Remove pre-abort transactions}
\label{reorder:alg:pre_abort}
\end{algorithm}


\begin{algorithm}
\KwData{A read-after-write dependency graph $G$ and a set $S$ of transactions}
\KwResult{A sequence $L$ of ordered transactions}
$heap\gets \emptyset$\;
\For {$t \in S$} {
	$heap.add(t)$\;
}
$L\gets \emptyset$\;
\While {$heap\neq\emptyset$} {
	\tcp{schedule a transaction with the least number of aborts}
	$v = heap.top()$\;
	\tcp{abort unscheduled transactions depending on $v$}
	\For {$u \in G.out\_edges(v)$} {
		$t = S.get(u)$\;
		\If {$t.dec == TBD$} {
			$t.dec = IN\_BATCH\_ABORT$\;
			$G.remove\_all\_edges(u)$\;
			$heap.remove(u)$\;
		}
	}
	\tcp{remove dependencies}
	$G.remove\_all\_edges(v)$\;
	$L.add(v)$\;
}
\caption{Greedy algorithm for max-commit policy}
\label{reorder:alg:greedy}
\end{algorithm}

\begin{algorithm}
\KwData{Dependency graph $G$, two transactions $t$ and $t'$}
\KwResult{Return $true$ if it is better to schedule $t$ before $t'$}
$e\gets G.outer\_edges(t)$\;
$e'\gets G.outer\_edges(t')$\;
\Return{$e.size() < e'.size()$}\;
\caption{Compare transactions for max-commit policy}
\label{reorder:alg:max_commit}
\end{algorithm}

\begin{algorithm}
\KwData{Dependency graph $G$, two transactions $t$ and $t'$}
\KwResult{Return $true$ if it is better to schedule $t$ before $t'$}
\If {$t.priority \neq t'.priority$} {
	\Return{$t.priority > t'.priority$}\;
}
$e\gets G.outer\_edges(t)$\;
$e'\gets G.outer\_edges(t')$\;
\Return{$e.size() < e'.size()$}\;
\caption{Compare transactions for priority policy}
\label{reorder:alg:priority}
\end{algorithm}

\begin{algorithm}
\KwData{Dependency graph $G$, two transactions $t$ and $t'$}
\KwResult{Return $true$ if it is better to schedule $t$ before $t'$}
\If {$notFair(t, t')$} {
	\Return{$false$}\;
}
$e\gets G.outer\_edges(t)$\;
$e'\gets G.outer\_edges(t')$\;
\Return{$e.size() < e'.size()$}\;
\caption{Compare transactions for fairness policy}
\label{reorder:alg:fairness}
\end{algorithm}

\section{Evaluation}\label{sec:experiments}
In our experimental evaluation, we wanted to understand the effect of batching and reordering at storage and validator, the performance of our validator reordering algorithms and policies, parallelizing transaction reordering at validator, and the impact of system configuration parameters. In particular, we asked the following questions:
\begin{enumerate}
\item\vspace{-.8em} How well do our validator reordering algorithms from Section~\ref{subsec:validator_reordering:algorithm} perform? How does batching and reordering at validator using these algorithms affect the end-to-end system performance?
\item\vspace{-.8em} How does the batch size impact performance? 
\item\vspace{-.8em} How does parallel reordering impact the system performance?
\item\vspace{-.8em} How does storage and validator batching and reordering affect the system performance?
\item\vspace{-.8em} How do the different policies presented in Section~\ref{subsec:validator_reordering:policy} impact the system performance?
\item\vspace{-.8em} How does batching and reordering perform on real workloads?
\end{enumerate}

\subsection{Experimental Setup}
\label{subsec:experiment:implementation}

%Our system architecture consists of four components: a transaction generator, a processor thread, a storage thread, and a validator thread. The threads communicate through queues of requests; that is, there is a generator queue, a processor queue, a storage queue, and a validator queue.

Our system architecture consists of four components: transaction generation, a processor, storage, and validation. The components communicate through consumer-producer queues.

The transaction generator continuously produces new transactions into the system until the system reaches the maximum permitted concurrency level. The processor multiplexes transactions as a transaction coordinator, receives transaction requests from the transaction generator, sends read/write requests to the storage, sends validation requests to the validator, and replies to the transaction generator. It also restarts aborted transactions; thus, it only communicates commit decisions to the transaction generator. 
The storage continuously processes read and write requests. When batching is enabled, it buffers a batch of requests. When it processes a batch, it executes the optimal strategy that we discussed in Section~\ref{sec:overview}: It first executes all the write requests in the batch (discarding a write if a newer version exists in the storage), and then all the read requests. 

% validator
The validator performs backward validation. It receives the keys and versions of the reads and the keys of the writes in each transaction. It caches the write keys of committed transactions for future validations in an in-memory hash table, until these writes are overwritten by later updates. 
When batching is enabled, the validator collects the requests into a batch as they arrive, and runs one of the algorithms from Section~\ref{sec:validator_reordering} to determine a serialization order. Every transaction that passes validation is assigned an integer \emph{commit timestamp}, which corresponds to the version number of the updates it will install in the storage. 

% Validator
We have further decoupled the validator component into four subcomponents as described in Section~\ref{subsec:validator_reordering:parallel}. A batch preparation worker receives validation requests from the processor, packages transactions into batches, and queues them for reordering. A transaction reordering worker takes a batch from the queue, pre-validates and filters the transactions in the batch against the validator cache, reorders the transactions, and queues the ordered transactions as a new batch for validation. A validation worker takes a batch from the queue, serializes the transactions, and validates the transactions against the current validator cache, and sends committed transactions to the cache update worker. The cache update worker finally applies updates to the validator cache based on the transaction serialization order. 

% Details of validator implementation.
By default, the validator uses the sort-based greedy algorithm with the \texttt{prod-degree} policy and multi factor 2. When a batch comes, a dependency graph is created as described in Section~\ref{subsec:validator_reordering:algorithm}. We have observed that once the dependency graph has become very dense, the reordering at the validator is not beneficial. This is because the reordering algorithm takes longer, while less transactions commit due to the inherent higher contention level. In practice, we heuristically set a loose limit on the size of the dependency graph. Once we detect that the number of edges hits the limit during the construction of the dependency graph, we discard the graph and validates the transactions directly in their arrival order in the batch without reordering. 

% Parallelization.
We have parallelized the transaction generation, the storage, and the transaction reordering at the validator. By default, two transaction generators populate the transactions concurrently to supply sufficient load. Two storage workers concurrently process reads and writes, and the writes are applied based on its data versioning as described in Section~\ref{sec:overview}. In the validator, we first introduced pipeline parallelism by processing different subcomponents concurrently. Since we observed that transaction reordering is heavier weight as compared to other subcomponents, we increased its capacity by multi-threading. We used four transaction reordering workers by default.

% Hard ware and data population.
Our system is implemented in Java. All the experiments run on a machine with Intel Xeon E5-2630 CPU @2.20GHz and 16GB RAM. We use a key-value model for the storage, which we implement as an in-memory hash table. In our micro benchmark, we populate the database with 100K objects, each with an 8-byte key. The values are left null as they are not relevant to our evaluation. We generate a transactional workload where each transaction reads 5 objects and writes to 5 objects, with of the reads and one of the writes on the same object. The reads and writes are drawn from a Zipfian distribution, implemented based on the standard model by Gray et al.~\cite{gray1994quickly}. We limit the concurrency level to 300, i.e., at any time there are at most 300 live transactions in the system. The default batch size is 40 for both storage and validator.

% baseline
The baseline configuration represents the system running with both storage and validator batching turned off. We add a batch mode to separate the effect of batching and reordering. In the batch mode, batching is enabled at both storage and validator, but no reordering is performed. The batch mode can benefit from better caching with tighter loops in the processing. 

% other configuration
\eat{The validator uses the sort-based greedy algorithm with the \texttt{prod-degree} policy and multi factor 2.} 

% Misc
All our experimental figures show the averages of 10 runs, each lasting for 60 seconds in between a 10-second warm-up and a 10-second cool-down time. The standard deviation was not significant in any of the experiments, so we omit the error bars for clarity of presentation. We report the good throughput (the number committed transactions per second), the average latency, and the percentile latency.


% *******************
% * Figures
% *******************
\begin{figure*}[t]
    \centering
    \begin{minipage}[b]{0.32\linewidth}
        \centering
        \includegraphics[width=\textwidth]{./exp_fig/fvs/fvs}
        \vspace{-2em}
        \caption{Size of FVS per graph with different algorithms}
        \label{fig:fvs:fvs}
    \end{minipage}
    \begin{minipage}[b]{0.32\linewidth}
        \centering
        \includegraphics[width=\textwidth]{./exp_fig/fvs/latency}
        \vspace{-2em}
        \caption{Running time of finding FVS with different algorithms}
        \label{fig:fvs:latency}
    \end{minipage}
    \begin{minipage}[b]{0.32\linewidth}
        \centering
        \includegraphics[width=\textwidth]{./exp_fig/greedy/tps}
        \vspace{-2em}
        \caption{Throughput with SCC-based and sort-based greedy algorithms}
        \label{fig:greedy:tps}
    \end{minipage}
    \vspace{-1em}
\end{figure*}

\begin{figure*}[t]
    \centering
    \begin{minipage}[b]{0.32\linewidth}
	\centering
	\includegraphics[width=\textwidth]{./exp_fig/greedy/latency}
	\vspace{-2em}
	\caption{Average latency for greedy algorithms}
	\label{fig:greedy:latency}
	\end{minipage}
    \begin{minipage}[b]{0.32\linewidth}
        \centering
        \includegraphics[width=\textwidth]{./exp_fig/greedy/percent95_latency}
        \vspace{-2em}
        \caption{Percentile latency for greedy algorithms}
        \label{fig:greedy:p95}
    \end{minipage}
    \begin{minipage}[b]{0.32\linewidth}
            \centering
            \includegraphics[width=\textwidth]{./exp_fig/bsize/tps}
            \vspace{-2em}
            \caption{Throughput with various batch sizes}
            \label{fig:bsize:tps}
    \end{minipage}    
    \vspace{-1em}
\end{figure*}

\begin{figure*}[t]
    \centering
    \begin{minipage}[b]{0.32\linewidth}
    	\centering
    	\includegraphics[width=\textwidth]{./exp_fig/bsize/latency}
    	\vspace{-2em}
    	\caption{Average latency with various batch sizes}
    	\label{fig:bsize:latency}
    \end{minipage}
    \begin{minipage}[b]{0.32\linewidth}
	\centering
	\includegraphics[width=\textwidth]{./exp_fig/bsize/percent95_latency}
	\vspace{-2em}
	\caption{Percentile latency with various batch sizes}
	\label{fig:bsize:p95}
	\end{minipage}
    \begin{minipage}[b]{0.32\linewidth}
	\centering
	\includegraphics[width=\textwidth]{./exp_fig/reorder/tps}
	\vspace{-2em}
	\caption{Throughput with different number of reorder workers}
	\label{fig:reorder:tps}
	\end{minipage}    
    \vspace{-1em}
\end{figure*}

\begin{figure*}[t]
    \centering
	\begin{minipage}[b]{0.32\linewidth}
	\centering
	\includegraphics[width=\textwidth]{./exp_fig/reorder/latency}
	\vspace{-2em}
	\caption{Average latency with different number of reorder workers}
	\label{fig:reorder:latency}
	\end{minipage}    
    \begin{minipage}[b]{0.32\linewidth}
	\centering
	\includegraphics[width=\textwidth]{./exp_fig/reorder/percent95_latency}
	\vspace{-2em}
	\caption{Percentile latency with different number of reorder workers}
	\label{fig:reorder:p95}
	\end{minipage}    
	\begin{minipage}[b]{0.32\linewidth}
	\centering
	\includegraphics[width=\textwidth]{./exp_fig/basic/tps}
	\vspace{-2em}
	\caption{Throughput under workloads of Zipfian distribution}
	\label{fig:basic:tps}
	\end{minipage}    
    \vspace{-1em}
\end{figure*}

\begin{figure*}[t]
    \centering
    \begin{minipage}[b]{0.32\linewidth}
	\centering
	\includegraphics[width=\textwidth]{./exp_fig/basic/latency}
	\vspace{-2em}
	\caption{Average latency under workloads of Zipfian distribution}
	\label{fig:basic:latency}
	\end{minipage}
    \begin{minipage}[b]{0.32\linewidth}
	\centering
	\includegraphics[width=\textwidth]{./exp_fig/basic/percent95_latency}
	\vspace{-2em}
	\caption{Percentile latency under workloads of Zipfian distribution}
	\label{fig:basic:p95}
	\end{minipage}
   \begin{minipage}[b]{0.32\linewidth}
	\centering
	\includegraphics[width=\textwidth]{{{./exp_fig/load/Z0.7_tps}}}
	\vspace{-2em}
	\caption{Throughput with micro benchmark (skew factor 0.7)}
	\label{fig:load_z0.7:tps}
	\end{minipage}
    \vspace{-1em}
\end{figure*}


% hard transactions
\begin{figure*}[t]
    \centering
	\begin{minipage}[b]{0.32\linewidth}
	\centering
	\includegraphics[width=\textwidth]{./exp_fig/restart/tps}
	\vspace{-2em}
	\caption{Throughput with tail latency optimized policies}
	\label{fig:restart:tps}
	\end{minipage}
    \begin{minipage}[b]{0.32\linewidth}
	\centering
	\includegraphics[width=\textwidth]{./exp_fig/restart/latency}
	\vspace{-2em}
	\caption{Average latency with tail latency optimized policies}
	\label{fig:restart:abort}
	\end{minipage}
    \begin{minipage}[b]{0.32\linewidth}
	\centering
	\includegraphics[width=\textwidth]{./exp_fig/restart/percent100_latency}
	\vspace{-2em}
	\caption{Percentile latency with tail latency optimized policies}
	\label{fig:restart:p100}
	\end{minipage}
%    \begin{minipage}[b]{0.32\linewidth}
%	\centering
%	\includegraphics[width=\textwidth]{{{./exp_fig/load/Z0.7_tps}}}
%	\vspace{-2em}
%	\caption{Throughput with micro benchmark (skew factor 0.7)}
%	\label{fig:load_z0.7:tps}
%	\end{minipage}
%   \begin{minipage}[b]{0.32\linewidth}
%	\centering
%	\includegraphics[width=\textwidth]{{{./exp_fig/load/Z0.8_tps}}}
%	\vspace{-2em}
%	\caption{Throughput with micro benchmark (skew factor 0.8)}
%	\label{fig:load_z0.8:tps}
%	\end{minipage}
	\vspace{-1em}
\end{figure*}


%\begin{figure*}[t]
%    \centering
%    \begin{minipage}[b]{0.32\linewidth}
%	\centering
%	\includegraphics[width=\textwidth]{{{./exp_fig/load/Z0.7_latency}}}
%	\vspace{-2em}
%	\caption{Average latency with micro benchmark (skew factor 0.7)}
%	\label{fig:load_z0.7:latency}
%	\end{minipage}
%\end{figure*}

%\begin{figure*}[t]
%    \centering
%    \begin{minipage}[b]{0.32\linewidth}
%	\centering
%	\includegraphics[width=\textwidth]{{{./exp_fig/small_bank/Z0.7_tps}}}
%	\vspace{-2em}
%	\caption{Throughput with Small Bank benchmark (skew factor 0.7)}
%	\label{fig:small_bank_z0.7:tps}
%	\end{minipage}
%   \begin{minipage}[b]{0.32\linewidth}
%       \centering
%        \includegraphics[width=\textwidth]{{{./exp_fig/small_bank/Z0.7_latency}}}
%        \vspace{-2em}
%        \caption{Average latency with Small Bank benchmark (skew factor 0.7)}
%        \label{fig:small_bank_z0.7:latency}
%    \end{minipage}
%	 \begin{minipage}[b]{0.32\linewidth}
%	\centering
%	\includegraphics[width=\textwidth]{{{./exp_fig/small_bank/Z0.8_tps}}}
%	\vspace{-2em}
%	\caption{Throughput with Small Bank benchmark (skew factor 0.8)}
%	\label{fig:small_bank_z0.8:tps}
%	\end{minipage}
%    \vspace{-1em}
%\end{figure*}

\begin{figure*}[t]
	\centering
%	\begin{minipage}[b]{0.32\linewidth}
%	\centering
%	\includegraphics[width=\textwidth]{{{./exp_fig/load/Z0.8_latency}}}
%	\vspace{-2em}
%	\caption{Average latency with micro benchmark (skew factor 0.8)}
%	\label{fig:load_z0.8:latency}
%\end{minipage}
	 \begin{minipage}[b]{0.32\linewidth}
	\centering
	\includegraphics[width=\textwidth]{{{./exp_fig/small_bank/tps}}}
	\vspace{-2em}
	\caption{Throughput with Small Bank benchmark)}
	\label{fig:small_bank:tps}
	\end{minipage}
	\begin{minipage}[b]{0.32\linewidth}
	\centering
	\includegraphics[width=\textwidth]{{{./exp_fig/small_bank/latency}}}
	\vspace{-2em}
	\caption{Average latency with Small Bank benchmark}
	\label{fig:small_bank:latency}
	\end{minipage}
	\begin{minipage}[b]{0.32\linewidth}
	\centering
	\includegraphics[width=\textwidth]{{{./exp_fig/small_bank/percent95_latency}}}
	\vspace{-2em}
	\caption{Percentile latency with Small Bank benchmark}
	\label{fig:small_bank:p95}
	\end{minipage}
%	 \begin{minipage}[b]{0.32\linewidth}
%	\centering
%	\includegraphics[width=\textwidth]{{{./exp_fig/small_bank/Z0.9_tps}}}
%	\vspace{-2em}
%	\caption{Throughput with Small Bank benchmark (skew factor 0.9)}
%	\label{fig:small_bank_z0.9:tps}
%	\end{minipage}
%	\begin{minipage}[b]{0.32\linewidth}
%	\centering
%	\includegraphics[width=\textwidth]{{{./exp_fig/small_bank/Z0.9_latency}}}
%	\vspace{-2em}
%	\caption{Average latency with Small Bank benchmark (skew factor 0.9)}
%	\label{fig:small_bank_z0.9:latency}
%	\end{minipage}
%    \vspace{-1em}
\end{figure*}

% *******************
% * Experiments
% *******************
\subsection{Validator Reordering Algorithms}
We first investigate the performance of the feedback vertex set algorithms from Section~\ref{subsec:validator_reordering:algorithm} with a comparison of the raw performance of the algorithms, i.e., their accuracy and running time. We run the algorithms on graphs constructed as described in Section~\ref{sec:ibvr}, using our micro benchmarks. 

\eat{We first test the algorithms offline on the dependency graphs constructed at validator when running the system. Each dependency graph is constructed from a batch of transactions at validator, excluding non-viable transactions, i.e., we only use transactions that don't have inter-batch conflicts. We compare the averages of the size of the feedback vertex set and the running time per dependency graph.}

We test the SCC-based greedy algorithm with the \texttt{max-degree} ($greedy\_max$), \texttt{sum-degree} ($greedy\_sum$) and \texttt{prod-degree} policies ($greedy\_prod$). We also test the sort-based greedy algorithm $greedy\_sort$ (using the \texttt{prod-degree} policy for sorting and multi factor 2), and the hybrid algorithm $hybrid\_m$. The hybrid algorithm uses $greedy\_prod$ as a subroutine when the size of the SCC is larger than $m$, and switches to the brute force search otherwise. By increasing the threshold, we can progressively approximate the optimal solution. 

We test these algorithms against several baselines: $search$ is an accurate,
brute force search algorithm; $random$ is the SCC-based greedy algorithm which
removes a vertex at random from each SCC to break the cycle. For each graph,
$random\_3$ runs $random$ 3 times and returns the smallest FVS, mitigating the
effect of bad random choices.


Figure~\ref{fig:fvs:fvs} shows the average size of the feedback vertex set found by each algorithm. The brute force search algorithm is so slow that it cannot produce results once the skew factor increases beyond $0.7$ as the graphs become denser.
The $random$ baseline computes a FVS whose size is almost twice as large as the greedy and the hybrid algorithms. Running the random algorithm multiple times produces similar results. This confirms the theoretical results which show that finding a good FVS is hard. The greedy algorithms, on the other hand, produce very accurate results. The average size of the FVS is almost identical to that of the brute force search when the skew factor is no larger than $0.7$, and is very close to the best hybrid algorithm ($hybrid\_20$, i.e., one that uses the brute force search when the size of the SCC is no larger than 20). Among the greedy algorithms, $greedy\_prod$ is consistently the best, although the difference is small.

Figure~\ref{fig:fvs:latency} shows the running time of the algorithms. The running time of the hybrid algorithm depends on the threshold for switching to brute force search. Thus, $hybrid\_20$ and $hybrid\_15$ have a longer running time than other algorithms, while the running time of $hybrid\_10$ is comparable to the SCC-based algorithms. Each of the SCC-based algorithms ($greedy\_max$, $greedy\_sum$, $greedy\_prod$, $random$) has a similar running time. The random algorithm takes slightly longer than the greedy algorithms because it removes more nodes and thus requires more computation. The running time of $random\_3$ is three times that of $random$, since it runs the random algorithm three times. The sort-based greedy algorithm ($greedy\_sort$), while slightly less accurate than the SCC-based greedy algorithms, reduces 74\% of the running time of these algorithms. 

We compare the end-to-end performance of the best SCC-based algorithm ($greedy\_prod$) against the sort-based greedy algorithm. Figure~\ref{fig:greedy:tps} and~\ref{fig:greedy:latency} show the  throughput and the average latency of the system with $greedy\_prod$ ($srvc\hbox{-}g$) and $greedy\_sort$ ($srvc\hbox{-}gs$). In both cases, storage batching is enabled.\eat{, and the greedy algorithm policy is set to minimizing the number of conflicts, i.e., the size of the FVS.} The $baseline$ line shows the throughput with both storage and validator batching disabled. 

The two greedy algorithms have similar throughput when the skew is very low. However, $greedy\_prod$ degrades significantly when data skew increases. This is because while $greedy\_prod$ is slightly more accurate, it takes much longer to run. This increases transaction latency and leads to more conflicts, especially when the data contention is high. $greedy\_sort$ consistently gives the highest throughput over all the workloads for its high accuracy and low running time. 

Figure~\ref{fig:greedy:p95} shows transaction latency by percentile, i.e., the latency threshold for up to 95\% of the transactions. The tail latency of $greedy\_sort$ is much lower than that of the other two, which is consistent with the throughput data.

\eat{In summary, the sort-based greedy algorithm is much faster than the ``smarter'' algorithms and only slightly worse in terms of accuracy, resulting in the best end-to-end system performance. For this reason, all subsequent experiments use the sort-based greedy algorithm with a \texttt{prod-degree} policy unless otherwise specified.}


\section{Experimental Evaluation: Parallel Validator Reordering}
\label{sec:experiments:parallel}


We evaluate the benefits of parallelism in the validator. Since we have observed that the reordering of FVS is the most time consuming subcomponent in the validator, we increase the number of threads to parallelize batch reordering as described in Appendix~\ref{sec:parallel}. 
Figure~\ref{fig:reorder:tps} and~\ref{fig:reorder:latency} show the throughput
and the average latency with the number of reordering workers from $1$ to $4$
($w1$, $w2$, $w3$, $w4$). The performance improves significantly with more reordering workers when data skew is medium to high. With $4$ workers, the throughput increases by up to $2.6\times$, and the average latency reduces by up to $39\%$, as compared to the result with $1$ worker. Figure~\ref{fig:reorder:p95} shows the percentile transaction latency. With more reordering workers, more transactions are reordered concurrently, and the transaction queuing time at validator is reduced. With $4$ workers, the tail latency reduces by up to $41\%$.\eat{ The improvement is not linear since the bottleneck of the system shifts to other
components as we increase the capacity of reordering.}

%The system can further scale up once the capacity of other components is carefully engineered and scaled, which is out of scope for this paper.
\eat{
In summary, parallel reordering reduces the queuing time for transactions, which leads to better throughput, average latency, and percentile latencies. Since 3 reordering workers have provided most of the benefit of parallel reordering in our configuration, we set the number of reordering workers to 3 in the reminder of our experiments.}

\subsection{Batch Size}
In this experiment, we explore how the batch size affects system performance. 
Smaller batch sizes should give lower latency but they offer fewer opportunities for reordering, leading to more aborts. 
We configure the system to perform both storage and validator batching with batch sizes from $20$ to $80$ ($b20$, $b40$, $b60$, and $b80$), using the same batch size at storage and validator. As before, $baseline$ is the system with both types of batching turned off. 

Figure~\ref{fig:bsize:tps} and~\ref{fig:bsize:latency} shows the throughput and the average latency of the system with different batch sizes as data skew increases. As expected, the throughput first rises as we increase the size of the batch, and then degrades when the batch becomes too large. Batch size 40 gives the best throughput.

The percentile latency displays a similar pattern, as shown in Figure~\ref{fig:bsize:p95}. Again the best batch size is 40. However, using batching always gives higher throughput and a better latency profile than the baseline. Given the above results, a batch size of 40 appears optimal for our configuration; we use this batch size in the remainder of our experiments. 
\subsection{Storage and Validator Batching}
\label{subsec:experiment:batching}

Next, we perform a detailed analysis on the effects of storage and validator batching. We configure the system in several different modes: no batching ($baseline$), batching only ($batch$), storage batching ($sr$), validator batching only\eat{with the \texttt{prod-degree} policy that maximizes the number of commits }($vc$), and both storage and validator batching ($srvc$).


\eat{As explained in Section~\ref{sec:overview}, batching and reordering affect the abort rate by reducing inter-batch and intra-batch aborts. The number of inter-batch aborts is affected by system-wide transaction latency, the freshness of the transactions' reads, and their access patterns. Validator reordering reduces the number of intra-batch aborts; however, storage reordering can increase the number of such aborts because it reduces inter-batch aborts (and thus more viable transactions end up in validator batches rather than aborting due to inter-batch conflicts). The overall throughput of the system is affected by both the transaction latency and the abort rate. }


Figure~\ref{fig:basic:tps},~\ref{fig:basic:latency}, and~\ref{fig:basic:p95} show the throughput, the average latency, and the percentile latency of different system modes under a variety of data skew parameters. 

Overall, using batching at storage and/or validator consistently leads to significant improvements in the throughput and the latency profiles over the baseline. Batching alone ($batch$) improves the throughput significantly due to lower amortized overhead per transaction and better caching with tighter loops in processing. In addition, storage reordering and validator reordering consistently further improve the throughput. Moreover, validator reordering significantly reduces the average latency and the percentile latencies.

\eat{When batching is enabled ($sr$ and $srvc$), the throughput is 2.1x-2.4x that of the baseline ($baseline$). In addition, using validator batching always gives a better latency profile (Figure~\ref{fig:basic:p95}).}

\eat{When the data contention is very low, the abort rate is low. Thus, the storage-batching-only mode ($sr$) gives similar throughput compared with $srvc$.  As the data skew increases, so does the number of intra-batch conflicts and aborts; the overhead of validator batching starts to pay off. In a medium-contention setting, using both validator and storage batching ($srvc$) gives the best throughput.}
When the data contention is extremely high, the number of intra-batch conflicts
that cannot be resolved by validator reordering increases. Validator reordering
is slower due to denser graphs, while bringing less benefit. Thus, the best throughput is achieved by using storage batching only ($sr$). 

We conducted additional experiments with the batch size fixed to evaluate the system's peak performance with the load varied. Figure~\ref{fig:load_z0.7:tps} shows the throughput of the system with batch size 20 and skew factor 0.7. The throughput increases with the concurrency level of the system, and then degrades as the system is overloaded. Enabling both storage and validator batching consistently outperforms the others. The figures on additional metrics and parameters are omitted due to the space limit.

\eat{
To summarize, it is always beneficial to enable storage batching since this technique reduces inter-batch aborts at a minimal cost. While validator batching consistently gives a percentile latency, it is most effective in mid-contention settings, when the reduction of intra-batch conflicts that it brings is sufficient to justify its cost. }

\subsection{Reducing Tail Latency}
\label{subsec:experiment:policy}

In this experiment, we explore validator reordering with more sophisticated policies as discussed in Section~\ref{subsec:validator_reordering:policy}.
%\eat{; specifically, we look at policies that aim to reduce the transaction tail latency}. Specifically, we explore the possibility of reducing transaction tail latency with latency-specific policies.
Our baselines are the \texttt{prod-degree} policy that maximizes the number of commits ($mc$) as well as no batching ($base$) and batching without reordering (\changed{$batch$}). 
Our first tail-latency aware policy ($rct$) favors transactions that have already been aborted and restarted. When choosing a node to include in the FVS, it chooses the node with the smallest number of restarts, breaking ties using \texttt{prod-degree}.
Our second latency-aware policy ($rdeg$) combines the number of restarts and the incoming/outgoing degrees of a transaction into a weight. It computes the weight of a node as
the product of in-degree and out-degree over the exponential of the number of
restarts with base 2. When choosing a node to include in the FVS, it picks the node with the highest weight. Thus, a node with a high degree product can have its weight reduced if the corresponding transaction has restarted many times.
Figures~\ref{fig:restart:tps} and~\ref{fig:restart:latency} show the throughput
and the average latency. The impact of tail-latency aware policies on transaction throughput and average latency are negligible as compared to when we maximize the number of commits ($mc$).
Figure~\ref{fig:restart:p100} shows the 
tail latencies above 90\%. 
While our first latency-aware policy $rct$ performs similar to $mc$, the more sophisticated policy $rdeg$ consistently performs significantly better than all the others, and it reduces the tail latency by up to 86\%.
\cut{
Thus, with latency-aware policies, we effectively reduce transaction tail latencies without sacrificing either the throughput or the average latency.
}

\subsection{End-to-End Performance on Benchmarks}
\label{subsec:experiment:end2end}
In our final experiment, we explore the end-to-end performance of batching in a realistic setting where batch size is fixed. We use two workloads: a micro benchmark and the Small Bank benchmark~\cite{alomari2008icde}. In our micro benchmark,  we generate the transactions as described in Section~\ref{subsec:experiment:implementation}. \eat{We introduce skewed accesses to the data where each object is drawn from Zipfian distribution.} The Small Bank benchmark contains transactions with a realistic and diverse combination of read and write conflicts. The transactions come from the financial domain: compute the balance of a customer's checking and savings accounts, deposit money to a checking account, transfer money from a checking account to a savings account, move all funds from one customer to another customer, and withdraw money from a customer's account. We use a Zipfian distribution to simulate skewed data accesses. We populate the database with 100K customers, i.e., 100K checking and 100K savings accounts. We use a batch size of 10 transactions and we vary the system concurrency level from 20 to 140 transactions. We simulate high data contention by introducing Zipfian skew factor (0.7 for the micro benchmark and 0.9 for the Small Bank benchmark, which has shorter transactions).
\eat{, i.e., the limit of active transactions in the system}


\eat{On the micro benchmark, Figure~\ref{fig:load:tps} shows the throughput at skew factor 0.7.} 
On the micro benchmark, Figure~\ref{fig:load:tps} shows the throughput. 
Using batching doubles the throughput as compared to the baseline, both for a given load and when considering the peak throughput over different loads. When the load is moderate, storage batching by itself performs best. As the load increases and the transactions become more conflict-prone, the benefit of validator batching outweighs its cost. This confirms our observation in Section~\ref{subsec:experiment:batching}. 

Figure~\ref{fig:load:latency} shows the average transaction latency. Both storage and validator batching reduce the latency as compared to the baseline. In addition, validator batching always reduces latency regardless of whether storage batching is enabled, again confirming our findings in Section~\ref{subsec:experiment:batching}.

\eat{Figures~\ref{fig:small_bank:tps} and~\ref{fig:small_bank:latency} show the throughput and latency of the system on the Small Bank benchmark at skew factor 0.9. }
Figures~\ref{fig:small_bank:tps} and~\ref{fig:small_bank:latency} show the throughput and latency of the system on the Small Bank benchmark. 
The performance impacts of batching are similar to those on the micro benchmark.
We ran additional experiments on both benchmarks varying the data skew; the results were similar to those shown and are omitted due to space limitations.
 % include load and small bank experiments.

\subsection{Experiment Summary}
We have conducted a set of experiments to understand the effect of batching and reordering at storage and validator, the performance of different reordering algorithms and policies, parallelizing transaction reordering at the validator, and the impact of system configuration parameters.

From the experiment results, we observed that 
\vspace{-1em}
\begin{enumerate}
\item The simple sort-based greedy algorithm for finding the feedback vertex set strikes a balance between accuracy and time complexity, and leads to the best overall system performance. 
\vspace{-.8em}
\item There is a sweet spot for the batch size, where the system achieves its best performance for the throughput, the average latency, and the percentile latency. We empirically selected the best batch size for our configuration and workload based on the experiment results.
\vspace{-.8em}
\item As we increase the level of parallelism in validator reordering, the throughput, the average latency, and the percentile latency all improve, especially when the data contention is medium to high.
\vspace{-.8em}
\item Batching alone improves throughput significantly. Adding ordering on top of batching consistently improves throughput, and significantly reduces the average and the percentile latency. It is always beneficial to enable storage reordering. While validator reordering consistently improves the percentile transaction latency, it sometimes hurt the throughput and latency when the data contention is at extremes.
\vspace{-.8em}
\item For alternative reordering policies at the validator, privileging transactions with a metric that combines the degree of the transaction in the dependency graph and its restart time significantly reduces the tail latency.
\vspace{-.8em}
\item Finally, in Small Bank benchmark, batching and reordering provides significantly better performance in the throughput, the average latency, and the percentile latency as compared to the baseline. 
\vspace{-.8em}
\end{enumerate}  
\section{Related Work}\label{sec:relwork}
\bailu{TODO: Check the related work from VLDB reviewers}

We discussed OCC and its applications in Section~\ref{sec:intro}. 
\eat{A detailed discussion on related work for the feedback vertex set problem can be found in~\cite{ding2016tr}.} 
Here, we discuss related work on concurrency control under high data contention, transaction scheduling, and batching. \eat{, and data clustering techniques.}

%\vspace{-1em}
{\bf High contention concurrency control.}
% data contention is bad for CC
The performance of concurrency control protocols suffers when either concurrency level and/or data contention are high~\cite{franaszek1985limitations}; this has particular impact on OCC~\cite{agrawal1987concurrency}. Hybrid approaches combine OCC and locking to limit the number of transaction restarts~\cite{thomasian1998distributed,yu1992analysis}. The problem can also be addressed by adjusting the concurrency level adaptively, limiting the number of arriving transactions and/or using an exponential backoff for aborted transactions~\cite{helal1993adaptive}. Transaction chopping reduces contention by partitioning transactions into smaller pieces and executing dependent pieces in a chained manner~\cite{mu2014extracting,shasha1995transaction,xie2015high}. Follow-up work explores other ways to analyze the access patterns of transactions to expose intermediate transaction state at a fine-grained level~\cite{wang2016scaling}. It is also possible to reduce conflicts by executing transactions at heterogeneous isolation levels~\cite{xie2014salt,xie2015high} or using a mix of optimistic and pessimistic concurrency control protocols~\cite{wang2016mostly}. While we also address the problem of reducing conflicts under data contention, our batching and reordering techniques are different from and complementary to previous work.

%\vspace{-1em}
%\paragraph{Transaction scheduling}
{\bf Transaction scheduling.}
The dynamic timestamp assignment technique assigns each transaction a timestamp interval and flexibly picks the commit timestamp from the interval~\cite{bayer1982dynamic}. A similar technique can be used to optimize read-only transactions in distributed asynchronous OCC~\cite{ding2015centiman}. This approach can be extended to dynamically update the timestamp intervals of live transactions while committing a different transaction~\cite{boksenbaum1987concurrent}. Dynamic timestamp assignment is compatible with our batching.

Transaction scheduling has also been studied in real-time databases, where urgent or high value transactions are prioritized~\cite{haritsa1993value}.
OCC with forward validation enables the validator to choose what to abort or defer a transaction if it would cause live transactions with higher priority to abort~\cite{haritsa1990dynamic, lam1995real,lee1993using}. There are also systems that use locking and preemption~\cite{abbott1992scheduling}, as well as hybrid optimistic/pessimistic methods~\cite{huang1991experimental, lin1990concurrency}. These approaches can be viewed as a simplified version of our validator reordering; moreover, none of the systems uses batching. 
Transactions can be batched and serialized before execution~\cite{thomson2012calvin, mu2014extracting, faleiro2014rethinking}. These are complementary to our work. Our approach is more flexible as it allows reordering at multiple stages in transaction execution.

%\vspace{-1em}
%\paragraph{Batching}
{\bf Batching.}
Batching to amortize costs and condense work is a common optimization technique. One application is to pack networking and logging messages~\cite{castro2002practical,ding2015centiman,friedman1997packing,glendenning2011scalable}. Batching is also widely applied to aggregate application requests to improve performance, including group commits~\cite{debrabant2013anti,hagmann1987reimplementing}, condensing IO requests~\cite{debrabant2013anti, faleiro2014lazy}, and Paxos~\cite{santos2012tuning}.
Since batching is often associated with a throughput/latency tradeoff, there is work on adaptive batching~\cite{friedman2006adaptive, nagle1984congestion}.
Those uses of batching are low-level and are not aware of the overall system infrastructure or the application semantics. Our work embraces batching as a core design principle at multiple stages of transaction execution. In addition, unlike previous work, we focus on the use of batching for reordering.


\section{Conclusions and Future Work}\label{sec:conclusion}
We have shown how to improve transaction performance in an OCC system by integrating batching and reordering into the system architecture at storage and validator. Batching allows the reordering of requests and reduces the number of aborts. We have formulated validator reordering as the problem of finding the minimal feedback vertex set (FVS) in a directed graph, and we have proposed two greedy algorithms for finding a FVS that are flexible and that perform well in practice. We have carried out an extensive experimental study in a main memory transaction processing prototype. Our experiments show that there is a sweet spot for the batch size for the best balance between latency and the flexibility of reordering. We have also investigated the impact of storage and validator batching on system performance. While both storage and validator batching consistently improve the throughput, validator reordering significantly reduces the latency profiles. We further proposed a parallel validator design. Our experiments showed that parallel reordering has improved both throughput and latency.


In future work, we plan to explore more sophisticated batch creation techniques, as well as systematically investigating adaptive batching to intelligently enable batching and adjust batch sizes for best system performance. In addition, we also want to explore
the opportunity to incorporate our batching and reordering techniques into existing open source transaction processing kernels.


\bibliographystyle{abbrv}
\bibliography{reference}

\end{document}
