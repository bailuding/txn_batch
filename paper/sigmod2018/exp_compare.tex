\subsection{Degradation Under Data Contention}
\label{subsec:experiment:compare}

We start off by comparing how the system performance degrades under data contention with other systems. We compare our implementation with two start-of-the-art systems: DBMS-X and Silo~\cite{tu2013speedy}. DBMS-X is a commercial database which offers a high performance, in-memory OLTP engine. Silo is a high-performance OLTP engine optimized for multi-core processors. Each engine uses a different variation of optimistic concurrency control as its concurrency control protocol. We choose the two engines because they are among the state-of-the-art general purpose OLTP engines with optimistic concurrency control.

We choose Small Bank benchmark~\cite{alomari2008icde} as the workload for comparison. 
The Small Bank benchmark contains transactions with a realistic and diverse
combination of read and write conflicts. The transactions come from the
financial domain: compute the balance of a customer's checking and savings
accounts, deposit money to a checking account, transfer money from a checking
account to a savings account, move funds from one customer to another, and withdraw money from a customer's account.

The Small Bank benchmark allows a fine-grained control of data contention. We use a Zipfian distribution to simulate skewed data accesses. We populate the database with 100K customers, i.e., 100K checking and 100K savings accounts.

We tune the performance of DBMS-X with the following optimizations: turning off logging to avoid disk IOs, batch executing transactions to reduce communication round-trips, and issuing transactions as stored procedures to reduce client-side overhead. We connect to the database with 10 clients, because our processor has 10 physical cores and it is the optimal number based on our empirical evaluation.

In this experiment, we focus on how the performance changes under data contention. Since we are interested in degradation of performance under high data contention, we show the relative trend of changes instead of absolute numbers. In fact, because the implementation of the three systems differs dramatically -- full-fledged commercial database, C++ implementation with low-level architectural optimization and Java implementation, it is also inappropriate to compare the raw numbers.

