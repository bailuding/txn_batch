\subsection{Degradation Under Data Contention}
\label{subsec:experiment:compare}

In our final experiment, we compare our implementation with Silo~\cite{tu2013speedy}. Silo is a high-performance OLTP engine optimized for multi-core processors using a variation of optimistic concurrency control as its concurrency control protocol. We choose Silo because it is among the state-of-the-art general purpose OLTP engines with optimistic concurrency control. In our implementation, we enable reordering at storage and validator ($srvr$) and with the policy to minimize the tail latencies ($srvr-rst$).

We focus on how the performance changes under data contention. Since we are interested in degradation of performance under high data contention, we show the relative trend of changes instead of absolute numbers. In fact, because the architecture and the implementation of the two systems differs dramatically -- C++ implementation with low-level optimization vs. Java implementation, comparing the raw numbers does not offer much insight. Throughput-wise, Silo is 4-30 times higher than our implementation.

Figure~\ref{fig:compare:z0.5_p100}, ~\ref{fig:compare:z0.7_p100}, and ~\ref{fig:compare:z0.9_p100} show the growth rate of the percentile latencies with various skew factors. We set the 50\% latency  and the abort rate with skew factor 0.5 as the baselines for computing the growth rate. As shown in the figure, the tail latency with our techniques grow much more elegantly as the data contention increases.
