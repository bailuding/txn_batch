\subsection{Parallel Validator Reordering}
In this experiment, we study the benefit of introducing parallelism into validator. Since we have observed that the reordering of the FVS is the most time consuming subcomponent in the validator, we increase the number of threads to parallelize batch reordering as described in Section~\ref{subsec:validator_reordering:parallel}. 

Figure~\ref{fig:reorder:tps} and~\ref{fig:reorder:latency} show the throughput
and the average latency with the number of reordering workers from 1 to 4
($rw\hbox{-}1$, $rw\hbox{-}2$, $rw\hbox{-}3$, $rw\hbox{-}4$). The performance improves significantly with more reordering workers when the skew factor is medium to high. With 4 reordering workers, the throughput is increased to up to 2.6x, and the average latency is reduced by up to 39\%, as compared to with 1 reordering workers. \eat{When the contention is fairly low, adding more reordering workers is not beneficial for the overhead it incurs.}Figure~\ref{fig:reorder:p95} shows the percentile transaction latency. With more reordering workers, more transactions are reordered concurrently, and the transaction queuing time at validator has reduced. With 4 reordering workers, the tail latency has reduced by up to 41\% as compared to with 1 reordering workers. 

Yet we didn't expect linear performance improvement since the bottleneck of the system may shift to other
components as we increase the capacity of the transaction reordering. The system can further scale up once the capacity
of other components is carefully engineered and scaled,
which is out of scope for this paper.
\eat{
In summary, parallel reordering reduces the queuing time for transactions, which leads to better throughput, average latency, and percentile latencies. Since 3 reordering workers have provided most of the benefit of parallel reordering in our configuration, we set the number of reordering workers to 3 in the reminder of our experiments.}
