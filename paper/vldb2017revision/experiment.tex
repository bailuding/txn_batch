\section{Evaluation}\label{sec:experiments}
In our experimental evaluation, we wanted to understand the effect of batching at storage and validator, the performance of our validator reordering algorithms and policies, parallelizing transaction reordering at validator, and the impact of system configuration parameters. In particular, we asked the following questions:
\begin{enumerate}
\item\vspace{-.5em} How well do our validator reordering algorithms from Section~\ref{subsec:validator_reordering:algorithm} perform? How does batching and reordering at validator using these algorithms affect the end-to-end system performance?
\item\vspace{-.5em} How does the batch size impact performance? 
%Using larger batches should give the validator and the storage more opportunities for reordering but it should also increase transaction latency, leading to more conflicts. 
\item\vspace{-.5em} How does parallel reordering impact the system performance?
\item\vspace{-.5em} How does storage and validator batching affect the system throughput, abort rate, and latency?
\item\vspace{-.5em} How do the different policies presented in Section~\ref{subsec:validator_reordering:policy} impact the system performance?
\item\vspace{-.5em} How does batching and reordering perform on micro-benchmarks and real workloads?
\end{enumerate}

\subsection{Experimental Setup}
\label{subsec:experiment:implementation}

%Our system architecture consists of four components: a transaction generator, a processor thread, a storage thread, and a validator thread. The threads communicate through queues of requests; that is, there is a generator queue, a processor queue, a storage queue, and a validator queue.

Our system architecture consists of four components: a transaction generator, a processor, storage, and a validator. The components communicate through request queues.

% of requests.
% that is, there is a generator queue, a processor queue, a storage queue, and a validator queue.
The transaction generator continuously produces new transactions into the system until the system reaches the maximum permitted concurrency level. The processor component multiplexes transactions, receives transaction requests from the transaction generator, sends read/write requests to the storage, sends validation requests to the validator, and replies to the transaction generator. It also restarts aborted transactions; thus, it only communicates commit decisions to the transaction generator. 
The storage worker continuously processes read and write requests. When batching is enabled, it buffers a batch of requests. When it processes a batch, it executes the optimal strategy that we discussed in Section~\ref{sec:overview}: It first executes all the write requests in the batch (discarding a write if a newer version exists in the storage), and then all the read requests. 

The validator performs backward validation. It receives the keys and versions of the reads and the keys of the writes in each transaction. It caches the write sets of committed transactions for future validations in a in-memory hash table, until these writes are overwritten by later committed transactions. 
When batching is enabled, the validator collects the requests into a batch as they arrive, and runs one of the algorithms from Section~\ref{sec:validator_reordering} to determine a serialization order. Every transaction that passes validation is assigned an integer \emph{commit timestamp}, which corresponds to the version number of the updates it will install in the storage.

% Implementation of FVS algorithm.
When a batch comes, a dependency graph is created as described in Section~\ref{subsec:validator_reordering:algorithm}. We have observed that once the dependency graph has become very dense, the reordering at the validator is not beneficial. This is because the reordering algorithm takes longer, while less transactions commit due to the inherent higher contention level. In practice, we heuristically set a loose limit on the size of the dependency graph. Once we detect that the number of edges hits the limit during the construction of the dependency graph, we discard the graph and validates the transactions directly in their arrival order in the batch without reordering.

% Validator
We have further decoupled the validator component into four subcomponent as described in Section~\ref{subsec:validator_reordering:parallel}. A batch preparation worker receives validation requests from the processor, packages transactions into batches, and queues them for reordering. A transaction reordering worker takes a batch from the queue, pre-validates a batch against the latest snapshot of the database in the validator cache, reorders the transactions in a batch, and queues them for validation. A validation worker takes a batch from the queue, serializes the transactions, and validates the transactions against the current validator cache, and sends committed transactions to the cache update worker. The cache update worker finally applies updates to the validator cache based on the transaction serialization order. 

% Parallelization.
We have parallelized the transaction generation, the storage, and the transaction reordering at the validator. Two transaction generators populate the transactions concurrently to supply sufficient load. Two storage workers concurrently process reads and writes, and the writes are applied based on its data versioning as described in Section~\ref{sec:overview}. In the validator, we first introduced pipeline parallelism by processing different subcomponents concurrently. Since we observed that transaction reordering is heavier weight as compared to other subcomponents, we increased its capacity by multi-threading. We used four transaction reordering workers by default.


Our system is implemented in Java. All the experiments run on a multicore machine, with Intel Xeon E5-2630 CPU @2.20GHz and 16GB RAM. We use a key-value model for the storage, which we implement as an in-memory hash table. In our micro benchmark, we populate the database with 100K objects, each with an 8-byte key. The values are left null as they are not relevant to our evaluation. We generate a transactional workload where each transaction reads 5 objects and writes to 5 objects, with one object appearing in both the read and the write sets. The reads and writes are drawn from a Zipfian distribution, which is implemented based on the standard model by Gray et al.~\cite{gray1994quickly}. We limit the concurrency level to 300, i.e., at any time there are at most 300 live transactions in the system. The default batch size is 40 for both storage and validator.

% other configuration
The validator uses the sort-based greedy algorithm with the \texttt{prod-degree} policy and multi factor 2. The baseline configuration represents the system running with both storage and validator batching turned off.  All our experimental figures show the averages of 10 runs, each lasting for 60 seconds in between a 10-second warm-up and a 10-second cool-down time. The standard deviation was not significant in any of the experiments, so we omit the error bars for clarity of presentation. We report the good throughput (the number committed transactions per second), the average latency, and the percentile latency.


% *******************
% * Figures
% *******************
\begin{figure*}[t]
    \centering
    \begin{minipage}[b]{0.32\linewidth}
        \centering
        \includegraphics[width=\textwidth]{./exp_fig/fvs/fvs}
        \vspace{-2em}
        \caption{Size of FVS per graph with different algorithms}
        \label{fig:fvs:fvs}
    \end{minipage}
    \begin{minipage}[b]{0.32\linewidth}
        \centering
        \includegraphics[width=\textwidth]{./exp_fig/fvs/latency}
        \vspace{-2em}
        \caption{Running time of finding FVS with different algorithms}
        \label{fig:fvs:latency}
    \end{minipage}
    \begin{minipage}[b]{0.32\linewidth}
        \centering
        \includegraphics[width=\textwidth]{./exp_fig/greedy/tps}
        \vspace{-2em}
        \caption{Throughput with SCC-based and sort-based greedy algorithms}
        \label{fig:greedy:tps}
    \end{minipage}
    \vspace{-1em}
\end{figure*}

\begin{figure*}[t]
    \centering
    \begin{minipage}[b]{0.32\linewidth}
	\centering
	\includegraphics[width=\textwidth]{./exp_fig/greedy/latency}
	\vspace{-2em}
	\caption{Average latency for greedy algorithms}
	\label{fig:greedy:latency}
	\end{minipage}
    \begin{minipage}[b]{0.32\linewidth}
        \centering
        \includegraphics[width=\textwidth]{./exp_fig/greedy/percent95_latency}
        \vspace{-2em}
        \caption{Percentile latency for greedy algorithms}
        \label{fig:greedy:p95}
    \end{minipage}
    \begin{minipage}[b]{0.32\linewidth}
            \centering
            \includegraphics[width=\textwidth]{./exp_fig/bsize/tps}
            \vspace{-2em}
            \caption{Throughput with various batch sizes}
            \label{fig:bsize:tps}
    \end{minipage}    
    \vspace{-1em}
\end{figure*}

\begin{figure*}[t]
    \centering
    \begin{minipage}[b]{0.32\linewidth}
    	\centering
    	\includegraphics[width=\textwidth]{./exp_fig/bsize/latency}
    	\vspace{-2em}
    	\caption{Average latency with various batch sizes}
    	\label{fig:bsize:latency}
    \end{minipage}
    \begin{minipage}[b]{0.32\linewidth}
	\centering
	\includegraphics[width=\textwidth]{./exp_fig/bsize/percent95_latency}
	\vspace{-2em}
	\caption{Percentile latency with various batch sizes}
	\label{fig:bsize:p95}
	\end{minipage}
    \begin{minipage}[b]{0.32\linewidth}
	\centering
	\includegraphics[width=\textwidth]{./exp_fig/reorder/tps}
	\vspace{-2em}
	\caption{Throughput with different number of reorder workers}
	\label{fig:reorder:tps}
	\end{minipage}    
    \vspace{-1em}
\end{figure*}

\begin{figure*}[t]
    \centering
	\begin{minipage}[b]{0.32\linewidth}
	\centering
	\includegraphics[width=\textwidth]{./exp_fig/reorder/latency}
	\vspace{-2em}
	\caption{Average latency with different number of reorder workers}
	\label{fig:reorder:latency}
	\end{minipage}    
    \begin{minipage}[b]{0.32\linewidth}
	\centering
	\includegraphics[width=\textwidth]{./exp_fig/reorder/percent95_latency}
	\vspace{-2em}
	\caption{Percentile latency with different number of reorder workers}
	\label{fig:reorder:p95}
	\end{minipage}    
	\begin{minipage}[b]{0.32\linewidth}
	\centering
	\includegraphics[width=\textwidth]{./exp_fig/basic/tps}
	\vspace{-2em}
	\caption{Throughput under workloads of Zipfian distribution}
	\label{fig:basic:tps}
	\end{minipage}    
    \vspace{-1em}
\end{figure*}

\begin{figure*}[t]
    \centering
    \begin{minipage}[b]{0.32\linewidth}
	\centering
	\includegraphics[width=\textwidth]{./exp_fig/basic/latency}
	\vspace{-2em}
	\caption{Average latency under workloads of Zipfian distribution}
	\label{fig:basic:latency}
	\end{minipage}
    \begin{minipage}[b]{0.32\linewidth}
	\centering
	\includegraphics[width=\textwidth]{./exp_fig/basic/percent95_latency}
	\vspace{-2em}
	\caption{Percentile latency under workloads of Zipfian distribution}
	\label{fig:basic:p95}
	\end{minipage}
    \begin{minipage}[b]{0.32\linewidth}
        \centering
        \includegraphics[width=\textwidth]{./exp_fig/restart/tps}
        \vspace{-2em}
        \caption{Throughput with tail latency optimized policies}
        \label{fig:restart:tps}
    \end{minipage}
    \vspace{-1em}
\end{figure*}


% hard transactions
\begin{figure*}[t]
    \centering
    \begin{minipage}[b]{0.32\linewidth}
	\centering
	\includegraphics[width=\textwidth]{./exp_fig/restart/latency}
	\vspace{-2em}
	\caption{Average latency with tail latency optimized policies}
	\label{fig:restart:abort}
	\end{minipage}
    \begin{minipage}[b]{0.32\linewidth}
	\centering
	\includegraphics[width=\textwidth]{./exp_fig/restart/percent100_latency}
	\vspace{-2em}
	\caption{Percentile latency with tail latency optimized policies}
	\label{fig:restart:p100}
	\end{minipage}
%    \begin{minipage}[b]{0.32\linewidth}
%	\centering
%	\includegraphics[width=\textwidth]{{{./exp_fig/load/Z0.7_tps}}}
%	\vspace{-2em}
%	\caption{Throughput with micro benchmark (skew factor 0.7)}
%	\label{fig:load_z0.7:tps}
%	\end{minipage}
   \begin{minipage}[b]{0.32\linewidth}
	\centering
	\includegraphics[width=\textwidth]{{{./exp_fig/load/Z0.8_tps}}}
	\vspace{-2em}
	\caption{Throughput with micro benchmark (skew factor 0.8)}
	\label{fig:load_z0.8:tps}
	\end{minipage}
\vspace{-1em}
\end{figure*}


%\begin{figure*}[t]
%    \centering
%    \begin{minipage}[b]{0.32\linewidth}
%	\centering
%	\includegraphics[width=\textwidth]{{{./exp_fig/load/Z0.7_latency}}}
%	\vspace{-2em}
%	\caption{Average latency with micro benchmark (skew factor 0.7)}
%	\label{fig:load_z0.7:latency}
%	\end{minipage}
%\end{figure*}

%\begin{figure*}[t]
%    \centering
%    \begin{minipage}[b]{0.32\linewidth}
%	\centering
%	\includegraphics[width=\textwidth]{{{./exp_fig/small_bank/Z0.7_tps}}}
%	\vspace{-2em}
%	\caption{Throughput with Small Bank benchmark (skew factor 0.7)}
%	\label{fig:small_bank_z0.7:tps}
%	\end{minipage}
%   \begin{minipage}[b]{0.32\linewidth}
%       \centering
%        \includegraphics[width=\textwidth]{{{./exp_fig/small_bank/Z0.7_latency}}}
%        \vspace{-2em}
%        \caption{Average latency with Small Bank benchmark (skew factor 0.7)}
%        \label{fig:small_bank_z0.7:latency}
%    \end{minipage}
%	 \begin{minipage}[b]{0.32\linewidth}
%	\centering
%	\includegraphics[width=\textwidth]{{{./exp_fig/small_bank/Z0.8_tps}}}
%	\vspace{-2em}
%	\caption{Throughput with Small Bank benchmark (skew factor 0.8)}
%	\label{fig:small_bank_z0.8:tps}
%	\end{minipage}
%    \vspace{-1em}
%\end{figure*}

\begin{figure*}[t]
	\centering
	\begin{minipage}[b]{0.32\linewidth}
	\centering
	\includegraphics[width=\textwidth]{{{./exp_fig/load/Z0.8_latency}}}
	\vspace{-2em}
	\caption{Average latency with micro benchmark (skew factor 0.8)}
	\label{fig:load_z0.8:latency}
\end{minipage}
	 \begin{minipage}[b]{0.32\linewidth}
	\centering
	\includegraphics[width=\textwidth]{{{./exp_fig/small_bank/Z0.7_tps}}}
	\vspace{-2em}
	\caption{Throughput with Small Bank benchmark (skew factor 0.7)}
	\label{fig:small_bank_z0.7:tps}
	\end{minipage}
	\begin{minipage}[b]{0.32\linewidth}
	\centering
	\includegraphics[width=\textwidth]{{{./exp_fig/small_bank/Z0.7_latency}}}
	\vspace{-2em}
	\caption{Average latency with Small Bank benchmark (skew factor 0.7)}
	\label{fig:small_bank_z0.7:latency}
	\end{minipage}
%	 \begin{minipage}[b]{0.32\linewidth}
%	\centering
%	\includegraphics[width=\textwidth]{{{./exp_fig/small_bank/Z0.9_tps}}}
%	\vspace{-2em}
%	\caption{Throughput with Small Bank benchmark (skew factor 0.9)}
%	\label{fig:small_bank_z0.9:tps}
%	\end{minipage}
%	\begin{minipage}[b]{0.32\linewidth}
%	\centering
%	\includegraphics[width=\textwidth]{{{./exp_fig/small_bank/Z0.9_latency}}}
%	\vspace{-2em}
%	\caption{Average latency with Small Bank benchmark (skew factor 0.9)}
%	\label{fig:small_bank_z0.9:latency}
%	\end{minipage}
%    \vspace{-1em}
\end{figure*}

% *******************
% * Experiments
% *******************
\subsection{Validator Reordering Algorithms}
We first investigate the performance of the feedback vertex set algorithms from Section~\ref{subsec:validator_reordering:algorithm} with a comparison of the raw performance of the algorithms, i.e., their accuracy and running time. We run the algorithms on graphs constructed as described in Section~\ref{sec:ibvr}, using our micro benchmarks. 

\eat{We first test the algorithms offline on the dependency graphs constructed at validator when running the system. Each dependency graph is constructed from a batch of transactions at validator, excluding non-viable transactions, i.e., we only use transactions that don't have inter-batch conflicts. We compare the averages of the size of the feedback vertex set and the running time per dependency graph.}

We test the SCC-based greedy algorithm with the \texttt{max-degree} ($greedy\_max$), \texttt{sum-degree} ($greedy\_sum$) and \texttt{prod-degree} policies ($greedy\_prod$). We also test the sort-based greedy algorithm $greedy\_sort$ (using the \texttt{prod-degree} policy for sorting and multi factor 2), and the hybrid algorithm $hybrid\_m$. The hybrid algorithm uses $greedy\_prod$ as a subroutine when the size of the SCC is larger than $m$, and switches to the brute force search otherwise. By increasing the threshold, we can progressively approximate the optimal solution. 

We test these algorithms against several baselines: $search$ is an accurate,
brute force search algorithm; $random$ is the SCC-based greedy algorithm which
removes a vertex at random from each SCC to break the cycle. For each graph,
$random\_3$ runs $random$ 3 times and returns the smallest FVS, mitigating the
effect of bad random choices.


Figure~\ref{fig:fvs:fvs} shows the average size of the feedback vertex set found by each algorithm. The brute force search algorithm is so slow that it cannot produce results once the skew factor increases beyond $0.7$ as the graphs become denser.
The $random$ baseline computes a FVS whose size is almost twice as large as the greedy and the hybrid algorithms. Running the random algorithm multiple times produces similar results. This confirms the theoretical results which show that finding a good FVS is hard. The greedy algorithms, on the other hand, produce very accurate results. The average size of the FVS is almost identical to that of the brute force search when the skew factor is no larger than $0.7$, and is very close to the best hybrid algorithm ($hybrid\_20$, i.e., one that uses the brute force search when the size of the SCC is no larger than 20). Among the greedy algorithms, $greedy\_prod$ is consistently the best, although the difference is small.

Figure~\ref{fig:fvs:latency} shows the running time of the algorithms. The running time of the hybrid algorithm depends on the threshold for switching to brute force search. Thus, $hybrid\_20$ and $hybrid\_15$ have a longer running time than other algorithms, while the running time of $hybrid\_10$ is comparable to the SCC-based algorithms. Each of the SCC-based algorithms ($greedy\_max$, $greedy\_sum$, $greedy\_prod$, $random$) has a similar running time. The random algorithm takes slightly longer than the greedy algorithms because it removes more nodes and thus requires more computation. The running time of $random\_3$ is three times that of $random$, since it runs the random algorithm three times. The sort-based greedy algorithm ($greedy\_sort$), while slightly less accurate than the SCC-based greedy algorithms, reduces 74\% of the running time of these algorithms. 

We compare the end-to-end performance of the best SCC-based algorithm ($greedy\_prod$) against the sort-based greedy algorithm. Figure~\ref{fig:greedy:tps} and~\ref{fig:greedy:latency} show the  throughput and the average latency of the system with $greedy\_prod$ ($srvc\hbox{-}g$) and $greedy\_sort$ ($srvc\hbox{-}gs$). In both cases, storage batching is enabled.\eat{, and the greedy algorithm policy is set to minimizing the number of conflicts, i.e., the size of the FVS.} The $baseline$ line shows the throughput with both storage and validator batching disabled. 

The two greedy algorithms have similar throughput when the skew is very low. However, $greedy\_prod$ degrades significantly when data skew increases. This is because while $greedy\_prod$ is slightly more accurate, it takes much longer to run. This increases transaction latency and leads to more conflicts, especially when the data contention is high. $greedy\_sort$ consistently gives the highest throughput over all the workloads for its high accuracy and low running time. 

Figure~\ref{fig:greedy:p95} shows transaction latency by percentile, i.e., the latency threshold for up to 95\% of the transactions. The tail latency of $greedy\_sort$ is much lower than that of the other two, which is consistent with the throughput data.

\eat{In summary, the sort-based greedy algorithm is much faster than the ``smarter'' algorithms and only slightly worse in terms of accuracy, resulting in the best end-to-end system performance. For this reason, all subsequent experiments use the sort-based greedy algorithm with a \texttt{prod-degree} policy unless otherwise specified.}


\section{Experimental Evaluation: Parallel Validator Reordering}
\label{sec:experiments:parallel}


We evaluate the benefits of parallelism in the validator. Since we have observed that the reordering of FVS is the most time consuming subcomponent in the validator, we increase the number of threads to parallelize batch reordering as described in Appendix~\ref{sec:parallel}. 
Figure~\ref{fig:reorder:tps} and~\ref{fig:reorder:latency} show the throughput
and the average latency with the number of reordering workers from $1$ to $4$
($w1$, $w2$, $w3$, $w4$). The performance improves significantly with more reordering workers when data skew is medium to high. With $4$ workers, the throughput increases by up to $2.6\times$, and the average latency reduces by up to $39\%$, as compared to the result with $1$ worker. Figure~\ref{fig:reorder:p95} shows the percentile transaction latency. With more reordering workers, more transactions are reordered concurrently, and the transaction queuing time at validator is reduced. With $4$ workers, the tail latency reduces by up to $41\%$.\eat{ The improvement is not linear since the bottleneck of the system shifts to other
components as we increase the capacity of reordering.}

%The system can further scale up once the capacity of other components is carefully engineered and scaled, which is out of scope for this paper.
\eat{
In summary, parallel reordering reduces the queuing time for transactions, which leads to better throughput, average latency, and percentile latencies. Since 3 reordering workers have provided most of the benefit of parallel reordering in our configuration, we set the number of reordering workers to 3 in the reminder of our experiments.}

\subsection{Batch Size}
In this experiment, we explore how the batch size affects system performance. 
Smaller batch sizes should give lower latency but they offer fewer opportunities for reordering, leading to more aborts. 
We configure the system to perform both storage and validator batching with batch sizes from $20$ to $80$ ($b20$, $b40$, $b60$, and $b80$), using the same batch size at storage and validator. As before, $baseline$ is the system with both types of batching turned off. 

Figure~\ref{fig:bsize:tps} and~\ref{fig:bsize:latency} shows the throughput and the average latency of the system with different batch sizes as data skew increases. As expected, the throughput first rises as we increase the size of the batch, and then degrades when the batch becomes too large. Batch size 40 gives the best throughput.

The percentile latency displays a similar pattern, as shown in Figure~\ref{fig:bsize:p95}. Again the best batch size is 40. However, using batching always gives higher throughput and a better latency profile than the baseline. Given the above results, a batch size of 40 appears optimal for our configuration; we use this batch size in the remainder of our experiments. 
\subsection{Storage and Validator Batching}
\label{subsec:experiment:batching}

Next, we perform a detailed analysis on the effects of storage and validator batching. We configure the system in several different modes: no batching ($baseline$), batching only ($batch$), storage batching ($sr$), validator batching only\eat{with the \texttt{prod-degree} policy that maximizes the number of commits }($vc$), and both storage and validator batching ($srvc$).


\eat{As explained in Section~\ref{sec:overview}, batching and reordering affect the abort rate by reducing inter-batch and intra-batch aborts. The number of inter-batch aborts is affected by system-wide transaction latency, the freshness of the transactions' reads, and their access patterns. Validator reordering reduces the number of intra-batch aborts; however, storage reordering can increase the number of such aborts because it reduces inter-batch aborts (and thus more viable transactions end up in validator batches rather than aborting due to inter-batch conflicts). The overall throughput of the system is affected by both the transaction latency and the abort rate. }


Figure~\ref{fig:basic:tps},~\ref{fig:basic:latency}, and~\ref{fig:basic:p95} show the throughput, the average latency, and the percentile latency of different system modes under a variety of data skew parameters. 

Overall, using batching at storage and/or validator consistently leads to significant improvements in the throughput and the latency profiles over the baseline. Batching alone ($batch$) improves the throughput significantly due to lower amortized overhead per transaction and better caching with tighter loops in processing. In addition, storage reordering and validator reordering consistently further improve the throughput. Moreover, validator reordering significantly reduces the average latency and the percentile latencies.

\eat{When batching is enabled ($sr$ and $srvc$), the throughput is 2.1x-2.4x that of the baseline ($baseline$). In addition, using validator batching always gives a better latency profile (Figure~\ref{fig:basic:p95}).}

\eat{When the data contention is very low, the abort rate is low. Thus, the storage-batching-only mode ($sr$) gives similar throughput compared with $srvc$.  As the data skew increases, so does the number of intra-batch conflicts and aborts; the overhead of validator batching starts to pay off. In a medium-contention setting, using both validator and storage batching ($srvc$) gives the best throughput.}
When the data contention is extremely high, the number of intra-batch conflicts
that cannot be resolved by validator reordering increases. Validator reordering
is slower due to denser graphs, while bringing less benefit. Thus, the best throughput is achieved by using storage batching only ($sr$). 

We conducted additional experiments with the batch size fixed to evaluate the system's peak performance with the load varied. Figure~\ref{fig:load_z0.7:tps} shows the throughput of the system with batch size 20 and skew factor 0.7. The throughput increases with the concurrency level of the system, and then degrades as the system is overloaded. Enabling both storage and validator batching consistently outperforms the others. The figures on additional metrics and parameters are omitted due to the space limit.

\eat{
To summarize, it is always beneficial to enable storage batching since this technique reduces inter-batch aborts at a minimal cost. While validator batching consistently gives a percentile latency, it is most effective in mid-contention settings, when the reduction of intra-batch conflicts that it brings is sufficient to justify its cost. }

\subsection{Reducing Tail Latency}
\label{subsec:experiment:policy}

In this experiment, we explore validator reordering with more sophisticated policies as discussed in Section~\ref{subsec:validator_reordering:policy}.
%\eat{; specifically, we look at policies that aim to reduce the transaction tail latency}. Specifically, we explore the possibility of reducing transaction tail latency with latency-specific policies.
Our baselines are the \texttt{prod-degree} policy that maximizes the number of commits ($mc$) as well as no batching ($base$) and batching without reordering (\changed{$batch$}). 
Our first tail-latency aware policy ($rct$) favors transactions that have already been aborted and restarted. When choosing a node to include in the FVS, it chooses the node with the smallest number of restarts, breaking ties using \texttt{prod-degree}.
Our second latency-aware policy ($rdeg$) combines the number of restarts and the incoming/outgoing degrees of a transaction into a weight. It computes the weight of a node as
the product of in-degree and out-degree over the exponential of the number of
restarts with base 2. When choosing a node to include in the FVS, it picks the node with the highest weight. Thus, a node with a high degree product can have its weight reduced if the corresponding transaction has restarted many times.
Figures~\ref{fig:restart:tps} and~\ref{fig:restart:latency} show the throughput
and the average latency. The impact of tail-latency aware policies on transaction throughput and average latency are negligible as compared to when we maximize the number of commits ($mc$).
Figure~\ref{fig:restart:p100} shows the 
tail latencies above 90\%. 
While our first latency-aware policy $rct$ performs similar to $mc$, the more sophisticated policy $rdeg$ consistently performs significantly better than all the others, and it reduces the tail latency by up to 86\%.
\cut{
Thus, with latency-aware policies, we effectively reduce transaction tail latencies without sacrificing either the throughput or the average latency.
}

\subsection{End-to-End Performance on Benchmarks}
\label{subsec:experiment:end2end}
In our final experiment, we explore the end-to-end performance of batching in a realistic setting where batch size is fixed. We use two workloads: a micro benchmark and the Small Bank benchmark~\cite{alomari2008icde}. In our micro benchmark,  we generate the transactions as described in Section~\ref{subsec:experiment:implementation}. \eat{We introduce skewed accesses to the data where each object is drawn from Zipfian distribution.} The Small Bank benchmark contains transactions with a realistic and diverse combination of read and write conflicts. The transactions come from the financial domain: compute the balance of a customer's checking and savings accounts, deposit money to a checking account, transfer money from a checking account to a savings account, move all funds from one customer to another customer, and withdraw money from a customer's account. We use a Zipfian distribution to simulate skewed data accesses. We populate the database with 100K customers, i.e., 100K checking and 100K savings accounts. We use a batch size of 10 transactions and we vary the system concurrency level from 20 to 140 transactions. We simulate high data contention by introducing Zipfian skew factor (0.7 for the micro benchmark and 0.9 for the Small Bank benchmark, which has shorter transactions).
\eat{, i.e., the limit of active transactions in the system}


\eat{On the micro benchmark, Figure~\ref{fig:load:tps} shows the throughput at skew factor 0.7.} 
On the micro benchmark, Figure~\ref{fig:load:tps} shows the throughput. 
Using batching doubles the throughput as compared to the baseline, both for a given load and when considering the peak throughput over different loads. When the load is moderate, storage batching by itself performs best. As the load increases and the transactions become more conflict-prone, the benefit of validator batching outweighs its cost. This confirms our observation in Section~\ref{subsec:experiment:batching}. 

Figure~\ref{fig:load:latency} shows the average transaction latency. Both storage and validator batching reduce the latency as compared to the baseline. In addition, validator batching always reduces latency regardless of whether storage batching is enabled, again confirming our findings in Section~\ref{subsec:experiment:batching}.

\eat{Figures~\ref{fig:small_bank:tps} and~\ref{fig:small_bank:latency} show the throughput and latency of the system on the Small Bank benchmark at skew factor 0.9. }
Figures~\ref{fig:small_bank:tps} and~\ref{fig:small_bank:latency} show the throughput and latency of the system on the Small Bank benchmark. 
The performance impacts of batching are similar to those on the micro benchmark.
We ran additional experiments on both benchmarks varying the data skew; the results were similar to those shown and are omitted due to space limitations.
 % include load and small bank experiments.

\subsection{Experiment Summary}
We have conducted a set of experiments to understand the effect of batching at storage and validator, the performance of different reordering algorithms and policies, parallelizing transaction reordering at the validator, and the impact of system configuration parameters.

From the experiment results, we observed that 
\vspace{-.5em}
\begin{itemize}
\item The simple sort-based greedy algorithm for finding the feedback vertex set strikes a balance between accuracy and time complexity, and leads to the best overall system performance. 
\vspace{-.5em}
\item There is a sweet spot for the batch size, where the system achieves its best performance for the throughput, average latency, and percentile latency. We empirically selected the best batch size for our configuration and workload based on the experiment results.
\vspace{-.5em}
\item As we increase the level of parallelism in validator reordering, the throughput, average latency, and percentile latency all improve, especially when the data contention is medium to high.
\vspace{-.5em}
\item It is always beneficial to enable storage batching. Validator batching consistently improves the percentile transaction latency, it can hurt the throughput and latency when the data contention is at extremes.
\vspace{-.5em}
\item For alternative reordering policies at the validator, privileging transactions with a metric that combines the degree of the transaction in the dependency graph and its restart time significantly reduces the tail latency.
\vspace{-.5em}
\item Finally, in both micro benchmark and Small Bank benchmark, reordering provides significantly better performance in the throughput, average latency, and percentile latency as compared to the baseline. 
\vspace{-.5em}
\end{itemize}  