\section{Conclusions and Future Work}\label{sec:conclusion}

We have shown how to significantly improve transaction performance in an OCC system through storage and validator batching and reordering. Besides clean problem formulations and
reducing validator reordering to the problem of finding the \changed{minimal} feedback vertex set (FVS), we proposed two new practical greedy algorithms for this problem
that are fast and that perform well in practice. We showed that our algorithms can integrate different policies into the reordering to optimize for \changed{a variety of performance objectives and system architectures, such as low tail latency and multi-threaded decentralized transaction processing}. We \changed{further} proposed a parallel validator design to reduce the overhead of reordering. Our extensive experimental study both in a prototype system, \changed{as well as with a state-of-the-art OLTP system and a commercial database system} show that both storage and validator batching consistently improve throughput, and that especially validator reordering 
significantly reduces latency profiles. We also demonstrated how we optimize for low tail latency with alternative policies, and how our parallelization further improves
throughput and reduced latency.

In future work, we plan to explore more sophisticated batch creation techniques. Since we observed a sweet spot for the  batch size in our experiment, we want to systematically investigate adaptive batching that intelligently adjusts the batch size for the best performance given the workload.\changed{We are also interested in designing additional policies to apply the idea of batching and reordering to alternative system architectures and concurrency control protocols.} \cut{As recent open source OLTP systems offer impressive throughput, we also plan to explore the opportunity to incorporate our techniques into other systems beyond systems that are based on OCC.}
