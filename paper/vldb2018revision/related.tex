\section{Related Work}\label{sec:relwork}
We discussed OCC and its applications in Section~\ref{sec:intro}. 
\eat{A detailed discussion on related work for the feedback vertex set problem can be found in~\cite{ding2016tr}.} 
Here, we discuss related work on concurrency control under high data contention, transaction scheduling, and batching. \eat{, and data clustering techniques.}

%\vspace{-1em}
{\bf High contention concurrency control.}
% data contention is bad for CC
The performance of concurrency control protocols suffers when either concurrency level and/or data contention are high~\cite{franaszek1985limitations, appuswamy17vldb}; this has particular impact on OCC~\cite{agrawal1987concurrency}. Hybrid approaches combine OCC and locking to limit the number of transaction restarts~\cite{thomasian1998distributed,yu1992analysis}. The problem can also be addressed by adjusting the concurrency level adaptively, limiting the number of arriving transactions and/or using an exponential backoff for aborted transactions~\changed{\cite{helal1993adaptive, lim2017cicada}}. Transaction chopping reduces contention by partitioning transactions into smaller pieces and executing dependent pieces in a chained manner~\cite{mu2014extracting,shasha1995transaction,xie2015high}. Follow-up work explores other ways to analyze the access patterns of transactions to expose intermediate transaction state at a fine-grained level~\cite{wang2016scaling}. It is also possible to reduce conflicts by executing transactions at heterogeneous isolation levels~\cite{xie2014salt,xie2015high} or using a mix of optimistic and pessimistic concurrency control protocols~\cite{wang2016mostly}.
While we also address the problem of reducing conflicts under data contention, our batching and reordering techniques are different from and complementary to previous work.

%\vspace{-1em}
%\paragraph{Transaction scheduling}
{\bf Transaction scheduling.}
The dynamic timestamp assignment technique assigns each transaction a timestamp interval and flexibly picks the commit timestamp from the interval~\cite{bayer1982dynamic}. A similar technique can be used to optimize read-only transactions in distributed asynchronous OCC~\cite{ding2015centiman}. This approach can be extended to dynamically update the timestamp intervals of live transactions while committing a different transaction~\cite{boksenbaum1987concurrent}.
\changed{Recent work also proposes lazy timestamp assignment to reduce conflicts~\cite{yu2016tictoc}.} Dynamic timestamp assignment is compatible with our batching.

Transaction scheduling has also been studied in real-time databases, where urgent or high value transactions are prioritized~\cite{haritsa1993value}.
OCC with forward validation enables the validator to choose what to abort or defer a transaction if it would cause live transactions with higher priority to abort~\cite{haritsa1990dynamic, lam1995real,lee1993using}. There are also systems that use locking and preemption~\cite{abbott1992scheduling}, as well as hybrid optimistic/pessimistic methods~\cite{huang1991experimental, lin1990concurrency}. These approaches can be viewed as a simplified version of our validator reordering; moreover, none of the systems uses batching. 
Transactions can be batched and serialized before execution~\cite{thomson2012calvin, mu2014extracting, faleiro2014rethinking}. These are complementary to our work. Our approach is more flexible as it allows reordering at multiple stages in transaction execution.

%\vspace{-1em}
%\paragraph{Batching}
{\bf Batching.}
Batching to amortize costs and condense work is a common optimization technique. One application is to pack networking and logging messages~\cite{castro2002practical,ding2015centiman,friedman1997packing,glendenning2011scalable}. Batching is also widely applied to aggregate application requests to improve performance, including group commits~\cite{debrabant2013anti,hagmann1987reimplementing}, condensing IO requests~\cite{debrabant2013anti, faleiro2014lazy}, and Paxos~\cite{santos2012tuning}.
Since batching is often associated with a throughput/latency tradeoff, there is work on adaptive batching~\cite{friedman2006adaptive, nagle1984congestion}.
Those uses of batching are low-level and are not aware of the overall system infrastructure or the application semantics. Our work embraces batching as a core design principle at multiple stages of transaction execution. In addition, unlike previous work, we focus on the use of batching for reordering.

